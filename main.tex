
\documentclass[10 pt]{amsart}

%\usepackage{etoolbox}
%\makeatletter
%\let\ams@starttoc\@starttoc
%\makeatother
%\makeatletter
%\let\@starttoc\ams@starttoc
%\patchcmd{\@starttoc}{\makeatletter}{\makeatletter\parskip\z@}{}{}
%\makeatother

%\usepackage[parfill]{parskip}

\usepackage[colorlinks=true,linkcolor=blue,citecolor=blue,urlcolor=blue]{hyperref}
\usepackage{bookmark}
\usepackage{amsthm,thmtools,amssymb,amsmath,amscd}

\usepackage{fancyhdr}
\usepackage{esint}
\bibliographystyle{/Users/J_Mac/Documents/Uni/TexTemplates/hamsplain}
\usepackage{enumerate}

\usepackage{pictexwd,dcpic}

\swapnumbers
\declaretheorem[name=Theorem,numberwithin=section]{thm}
\declaretheorem[name=Remark,style=remark,sibling=thm]{rem}
\declaretheorem[name=Lemma,sibling=thm]{lemma}
\declaretheorem[name=Proposition,sibling=thm]{prop}
\declaretheorem[name=Definition,style=definition,sibling=thm]{defn}
\declaretheorem[name=Corollary,sibling=thm]{cor}
\declaretheorem[name=Assumption,style=remark,sibling=thm]{ass}
\declaretheorem[name=Example,style=remark,sibling=thm]{example}
\declaretheorem[name=Notation,style=definition,sibling=thm]{notation}


\numberwithin{equation}{section}

\usepackage{cleveref}
\crefname{lemma}{Lemma}{Lemmata}
\crefname{prop}{Proposition}{Propositions}
\crefname{thm}{Theorem}{Theorems}
\crefname{cor}{Corollary}{Corollaries}
\crefname{defn}{Definition}{Definitions}
\crefname{example}{Example}{Examples}
\crefname{rem}{Remark}{Remarks}
\crefname{ass}{Assumption}{Assumptions}
\crefname{notation}{Notation}{Notation}



%Symbols
\renewcommand{\~}{\tilde}
\renewcommand{\-}{\bar}
\newcommand{\bs}{\backslash}
\newcommand{\cn}{\colon}
\newcommand{\sub}{\subset}

\newcommand{\N}{\mathbb{N}}
\newcommand{\Z}{\mathbb{Z}}
\newcommand{\Q}{\mathbb{Q}}
\newcommand{\R}{\mathbb{R}}
\newcommand{\C}{\mathbb{C}}
\renewcommand{\S}{\mathbb{S}}
\renewcommand{\H}{\mathbb{H}}
\newcommand{\K}{\mathbb{K}}
\newcommand{\Di}{\mathbb{D}}
\newcommand{\B}{\mathbb{B}}
\newcommand{\8}{\infty}

%Greek letters
\renewcommand{\a}{\alpha}
\renewcommand{\b}{\beta}
\newcommand{\g}{\gamma}
\renewcommand{\d}{\delta}
\newcommand{\e}{\epsilon}
\renewcommand{\k}{\kappa}
\renewcommand{\l}{\lambda}
\renewcommand{\o}{\omega}
\renewcommand{\t}{\theta}
\newcommand{\s}{\sigma}
\newcommand{\p}{\varphi}
\newcommand{\z}{\zeta}
\newcommand{\vt}{\vartheta}
\renewcommand{\O}{\Omega}
\newcommand{\D}{\Delta}
\newcommand{\G}{\Gamma}
\newcommand{\T}{\Theta}
\renewcommand{\L}{\Lambda}

%Mathcal Letters
\newcommand{\cL}{\mathcal{L}}
\newcommand{\cT}{\mathcal{T}}
\newcommand{\cA}{\mathcal{A}}
\newcommand{\cW}{\mathcal{W}}
\newcommand{\cH}{\mathcal{H}}
\newcommand{\cS}{\mathcal{S}}


%Mathematical operators
\newcommand{\INT}{\int_{\O}}
\newcommand{\DINT}{\int_{\d\O}}
\newcommand{\Int}{\int_{-\infty}^{\infty}}
\newcommand{\del}{\partial}
\newcommand{\de}[1]{\frac{\partial}{\partial #1}}
\newcommand{\n}{\nabla}
\newcommand{\II}[2]{\mrm{II}\br{#1,#2}}




\newcommand{\ip}[2]{\left\langle #1,#2 \right\rangle}
\newcommand{\fr}[2]{\frac{#1}{#2}}
\newcommand{\x}{\times}

\DeclareMathOperator{\dive}{div}
\DeclareMathOperator{\id}{id}
\DeclareMathOperator{\pr}{pr}
\DeclareMathOperator{\Diff}{Diff}
\DeclareMathOperator{\supp}{supp}
\DeclareMathOperator{\graph}{graph}
\DeclareMathOperator{\osc}{osc}
\DeclareMathOperator{\const}{const}
\DeclareMathOperator{\dist}{dist}
\DeclareMathOperator{\loc}{loc}
\DeclareMathOperator{\tr}{tr}
\DeclareMathOperator{\Rm}{Rm}
\DeclareMathOperator{\Rc}{Rc}
\DeclareMathOperator{\grad}{grad}


%Environments
\newcommand{\Theo}[3]{\begin{#1}\label{#2} #3 \end{#1}}
\newcommand{\pf}[1]{\begin{proof} #1 \end{proof}}
\newcommand{\eq}[1]{\begin{equation}\begin{alignedat}{2} #1 \end{alignedat}\end{equation}}
\newcommand{\IntEq}[4]{#1&#2#3	 &\quad &\text{in}~#4,}
\newcommand{\BEq}[4]{#1&#2#3	 &\quad &\text{on}~#4}
\newcommand{\br}[1]{\left(#1\right)}



%Logical symbols
\newcommand{\Ra}{\Rightarrow}
\newcommand{\ra}{\rightarrow}
\newcommand{\hra}{\hookrightarrow}
\newcommand{\mt}{\mapsto}

%Fonts
\newcommand{\mc}{\mathcal}
\renewcommand{\it}{\textit}
\newcommand{\mrm}{\mathrm}

%Spacing
\newcommand{\hp}{\hphantom}


\parindent 0 pt

%\protected\def\ignorethis#1\endignorethis{}
%\let\endignorethis\relax
%\def\TOCstop{\addtocontents{toc}{\ignorethis}}
%\def\TOCstart{\addtocontents{toc}{\endignorethis}}












\begin{document}

\title[Harnack inequalities for curvature flows]{Harnack inequalities for curvature flows in Riemannian and Lorentzian manifolds}

\maketitle

\section{Introduction}

Let $N=N^{n+1}$ be a Riemannian or Lorentzian manifold and let $M^{n}$ be a smooth and orientable manifold. Let $\s = 1$ in the Riemannian case and $\s = -1$ in the Lorentzian case. Let
\eq{x\cn M^{n}\x[0,T)\ra N}
be a family of strictly convex spacelike embeddings (in the Riemannian case just ignore the term ``spacelike''), which evolve by the curvature flow
\eq{\label{Flow}\dot{x}=-\s f\nu-x_{;k}b^{ki}f_{;i},}
where $\nu$ is a unit normal vector field along $M_{t}=x(M,t)$ (which satisfies $\s=\ip{\nu}{\nu}$ from the spacelike condition), $(b^{ij})$ is the inverse of the second fundamental form $(h_{ij}),$ indices appearing after a semicolon denote the components of the covariant derivative with respect to the induced metric $(g_{ij})$ and
\eq{f\cn \R_{+}\x UN \x \Sigma^{0,2}_{+}(M)\x\Sigma^{0,2}_{+}(M)\ra \R,}
where $\Sigma^{0,2}_{+}(M)\sub T^{0,2}M$ is the subbundle of positive definite tensors of type $(0,2)$, $UN$ is the unit sphere bundle on $N$, and $f$ is nonzero and invariant under parallel transport.
We assume that $f$ is of the form
\eq{f(s,\nu,h_{ij},g_{ij})=\p(s)F(\nu,h_{ij},g_{ij}),}
where $\p\in C^2(\R_+)$ and $s$ is the support function. For arbitrary backgrounds, suitable support functions do not exist, but it does no harm to include $\p$, simply taking $\p \equiv 1$ whenever we don't have a suitable support function.

The flow \eqref{Flow} is equivalent to,
\eq{\label{FlowStandard} \dot{\tilde{x}} = -\s \tilde{f} \tilde{\nu}}
by letting \(u_t\) denote the flow of the vector field $b^{ki}f_{;i}$ on $M$ and defining $\tilde{x} (\xi, t) = x (u_t(\xi), t)$. We call $\tilde{x}$ satisfying \eqref{FlowStandard} the \emph{standard parametrisation} as in \cite{Andrews:09/1994}.

Central to our approach is the flow \eqref{Flow} which is a reparametrisation of the flow as usually given in standard parametrisation \eqref{FlowStandard}. Let us note that in a Euclidean background, $N = \R^{n+1}$, one may consider the unit normal at time $t$, as the Gauss map $\nu_t : M \to \S^n$, which is a diffeomorphism whenever $x_t(M)$ is convex. The Gauss map parametrisation $y_t: \S^n \to \R^{n+1}$ \cite{Andrews:09/1994} is such that $\nu_t(y_t(z)) = z$ for all $z \in \S^n$ whence $\dot{\nu} = 0$. Furthermore, calculations may be performed with respect to the fixed, canonical, round metric $g_{\operatorname{can}}$. These two properties, a static metric and static normal provide immense benefit, not only in simplifying the generally long computations associated with differential Harnack inequalities, but also by lending insight into why such long computations yield such a simple, elegant differential Harnack inequality.

The Gauss map parametrisation just described is manifestly Euclidean, and given the utility of such a parametrisation, analogous results in other background spaces should be highly prized. The cornerstone of our approach is that the normal \(\nu\) is static in the parametrisation \eqref{Flow} and the time derivative of the induced metric \(g\) is only felt through the changing parametrisation, $x$. See \Cref{Ev-g-nu}, analogous to the Gauss map parametrisation, valid in arbitrary backgrounds.

Another crucial aspect of our approach is that it also addresses the question of how to determine the appropriate Harnack quantity. The philosophy put forward by Hamilton in \cite{Hamilton:/1995,Hamilton:/1993} is that equality should be attained on expanding solitons, just as equality in the Li-Yau Harnack inequality \cite{LiYau:/1986} is attained by the heat kernel, itself an expanding soliton. Thus Hamilton follows a procedure of differentiating the soliton equation to obtain soliton identities which eventually lead to the appropriate form for the Harnack quantity. In \cite{Andrews:09/1994}, Andrews showed that under the Gauss map parametrisation, one may consider the evolution of the support function to which the Li-Yau approach may be applied to determine the Harnack quantity. This is also effectively our approach, although for general backgrounds we don't have the luxury of a suitable support function, but have the great fortune of the reparametrisation \eqref{Flow}. Namely, we define
\begin{equation}
\label{eq:Q}
u = \frac{\dot{f}}{f}
\end{equation}
for the Harnack quantity. For shrinking flows (i.e. positive $f$) we have $u = \partial_t \ln f$ just as for Li-Yau \cite{LiYau:/1986} and \cite{Andrews:09/1994}. For expanding flows (i.e. negative $f$), we have the analogous formulation, $u = - \partial_t \ln (-f)$, taking into account the sign, and it seems simpler to use the form given in \eqref{eq:Q} for both cases. Then, transforming to the standard parametrisation, we find that,
\[
\tilde{u} = \frac{1}{\tilde{f}} \partial_t \left(f(u_t(x), t)\right) = \frac{1}{\tilde{f}} \left(\dot{\tilde{f}} - \tilde{b}^{ij} \tilde{f}_{;i} \tilde{f}_{;j}\right)
\]
which is precisely Hamilton's Harnack quantity.

The basic Harnack inequality we prove is the following:

\Theo{thm}{Harnack Inequality}{
Let $N$ be locally symmetric and with non-negative sectional curvature. Let $F$ be $1$-homogeneous, positive, and either convex or inverse concave if $N$ is flat and $f = F^p$. Then for a range of $p$ we have,
\[
u \geq -frac{p+1}{p} \frac{1}{t}
\]
for all $t > 0$.}

The range of admissible $p$ varies depending on the background space and the speed $f$ and is given expclicitly in each theorem below.

\section{Evolution equations}

\Theo{notation}{}{
The normal $\nu$ is supposed to be the same normal as in the Gaussian formula
\eq{x^{\a}_{;ij}:=\fr{\del^2 x^{\a}}{\del \xi^i\del \xi^j}+\-\G^{\a}_{\b\g}\fr{\del x^{\b}}{\del \xi^i}\fr{\del x^{\g}}{\del \xi^j}-\G^k_{ij}\fr{\del x^{\a}}{\del \xi^k}=-\s h_{ij}\nu^{\a}.}
Recall also the Weingarten equation,
\eq{\nu_{;i}=h^{k}_{i}x_{;k}.}
Geometric quantities of the ambient space are denoted with an overbar, e.g., $(\-g_{\a\b})$ for the metric and Greek indices range from $0$ to $n.$ Induced quantities are denoted for example by $(g_{ij})$, $(h_{ij}),$ where Latin indices range from $1$ to $n.$
Our definition of the $(1,3)$ Riemannian curvature tensor is given by
\eq{\mrm{Rm}(X,Y)Z=\n_{Y}\n_{X}Z-\n_{X}\n_{Y}Z-\n_{[Y,X]}Z} 
and the $(0,4)$ version is
\eq{\mrm{Rm}(W,X,Y,Z)=\ip{W}{\mrm{Rm}(X,Y)Z}.}
Hence the curvature tensor of the ambient space reads in coordinates
\eq{\-R^{\a}_{\b\g\d}= [\-R(\partial_{\delta}, \partial_{\gamma})\partial_{\beta}]^{\alpha} = \-\G^{\a}_{\b\d,\g}-\-\G^{\a}_{\b\g,\d}+\-\G^{\e}_{\b\d}\-\G^{\a}_{\e\g}-\-\G^{\e}_{\b\g}\-\G^{\a}_{\e\d},}
where $\-\G$ are the Christoffel symbols of the metric $\-g$ and a comma denotes ordinary differentiation.
We obtain the Ricci identities for differentiating a contravariant vector field $\eta=\eta^{\a}\del_{\a}\cn$
\eq{\eta^{\a}_{;\b\g}&=\br{\eta^{\a}_{;\b}}_{,\g}+\-\G^{\a}_{\d\g}\eta^{\d}_{;\b}-\-\G^{\d}_{\b\g}\eta^{\a}_{;\d}\\
                    &=\eta^{\a}_{,\b\g}+\-\G^{\a}_{\b\d,\g}\eta^{\d}+\-\G^{\a}_{\b\d}\eta^{\d}_{,\g}+\-\G^{\a}_{\d\g}\eta^{\d}_{,\b}+\-\G^{\a}_{\d\g}\-\G^{\d}_{\e\b}\eta^{\e}-\-\G^{\d}_{\b\g}\eta^{\a}_{,\d}-\-\G^{\d}_{\b\g}\-\G^{\a}_{\e\d}\eta^{\e}.}
In Riemannian normal coordinates:
\eq{\eta^{\a}_{;\b\g}=\eta^{\a}_{,\b\g}+\-\G^{\a}_{\b\d,
\g}\eta^{\d}&=\eta^{\a}_{,\g\b}+\-\G^{\a}_{\g\d,\b}\eta^{\d}+\br{\-\G^{\a}_{\b\d,\g}-\-\G^{\a}_{\g\d,\b}}\eta^{\d}\\
            &=\eta^{\a}_{;\g\b}+\-R^{\a}_{\d\g\b}\eta^{\d}=\eta^{\a}_{;\g\b}-\-R^{\a}_{\d\b\g}\eta^{\d}.}


There also holds
\eq{\-R_{\a\b\g\d}=\-g_{\a\e}\-R^{\e}_{\b\g\d}.}
Thus the Ricci tensor is given by
\eq{\-R_{\a\b}=\-R^{\g}_{\a\g\b}.}
}
\Theo{notation}{}{
We will use the following abbreviations to simplify the calculations. Set
\eq{\label{A1}A^{k}_{i}=\s fh^{k}_{i}+\br{b^{kl}f_{;l}}_{;i},}
then we also have
\eq{\label{A2}\dot{x}_{;i}&=-\s f_{;i}\nu-\s f h^{k}_{i}x_{;k}+\s h_{ki}b^{kj}f_{;j}\nu-x_{;k}\br{b^{kj}f_{;j}}_{;i}\\
            &=-A^{k}_{i}x_{;k}.}
Furthermore, we define (generally) non-symmetric bilinear forms,
\eq{B_{ij}=\fr{h_{ik}A^{k}_{j}}{f}}
and
\eq{\L_{ij}=\-R_{\a\b\g\d}\dot{x}^{\a}x^{\b}_{;i}x^{\g}_{;j}\nu^{\d}.}
}

The motivation for our particular reparametrisation is the fact that the normal does not move:

\Theo{lemma}{Ev-g-nu}{
Along the flow \eqref{Flow} there hold
\eq{\dot{\nu}=0}
and
\eq{\label{Ev-g}\dot{g}_{ij}=-g_{ik}A^{k}_{j}-g_{jk}A^{k}_{i}.}
}

\pf{
We have $\ip{\dot{\nu}}{\nu}=0$ and due to \eqref{A2} there holds $\ip{\dot{\nu}}{x_{;i}}=-\ip{\nu}{\dot{x}_{;i}}=0.$
Furthermore
\eq{\dot{g}_{ij}=\del_{t}\ip{x_{;i}}{x_{;j}}=-g_{ik}A^{k}_{j}-g_{jk}A^{k}_{i}.}
}

Also, the evolution of the second fundamental form simplifies.

\Theo{lemma}{Ev-W-h}{
Along the flow \eqref{Flow} there holds
%\eq{\label{Ev-W}\dot{h}^{i}_{j}=h^{k}_{j}A^{i}_{k}+\-R_{\a\b\g\d}\dot{x}^{\a}x^{\b}_{;j}x^{\g}_{;m}\nu^{\d}g^{mi}}
%and
\eq{\label{Ev-h}\dot{h}_{ij}=-fB_{ij}+\L_{ij}.}
}





\pf{
Let $\fr{D}{dt}$ denote the covariant derivative of a vector field in $N$ along the curve $x(\xi,\cdot)$ for fixed $\xi\in M.$ Due to the Ricci identities and the Weingarten equation we have
\eq{\dot{h}^{k}_{j}x^{\a}_{;k}+h^{k}_{j}\dot{x}_{;k}^{\a}=\fr{D}{dt}\br{\nu^{\a}_{;j}}=\br{\dot{\nu}^{\a}}_{;j}-\bar{R}^{\a}_{\b\g\d}\nu^{\b}x^{\g}_{;j}\dot{x}^{\d}=-\bar{R}^{\a}_{\b\g\d}\nu^{\b}x^{\g}_{;j}\dot{x}^{\d}}
and hence, using \eqref{A2} and multiplying both sides with $\-g_{\a\b}x^{\b}_{;m},$
\eq{\label{Ev-W-h-1}\dot{h}^{k}_{j}g_{km}=h^{k}_{j}A^{l}_{k}g_{lm}-\-R_{\a\b\g\d}x^{\a}_{;m}\nu^{\b}x^{\g}_{;j}\dot{x}^{\d}.}
Now \eqref{Ev-h} follows from \eqref{Ev-g} and \eqref{Ev-W-h-1}
\eq{\dot{h}_{ij}=\dot{h}^{k}_{j}g_{ki}+h^{k}_{j}\dot{g}_{ki}&=h^{k}_{j}A^{l}_{k}g_{li}-\-R_{\a\b\g\d}x^{\a}_{;i}\nu^{\b}x^{\g}_{;j}\dot{x}^{\d}-h^{k}_{j}g_{kr}A^{r}_{i}-h^{k}_{j}g_{il}A^{l}_{k}\\
            &=-h_{rj}A^{r}_{i}-\-R_{\a\b\g\d}x^{\a}_{;i}\nu^{\b}x^{\g}_{;j}\dot{x}^{\d}.}
We have to swap $i$ and $j$ in the first term of the right hand side. Due to the Codazzi equation and the symmetries of the Riemannian curvature tensor we have
\eq{-h_{rj}A^{r}_{i}&=-\s f h_{rj}h^{r}_{i}-h_{rj}\br{b^{rl}f_{;l}}_{;i}\\
            &=-\s fh_{ri}h^{r}_{j}-f_{;ij}+h_{rj;i}b^{rl}f_{;l}\\
            &=-\s fh_{ri}h^{r}_{j}-f_{;ij}+h_{ri;j}b^{rl}f_{;l}+\-R_{\a\b\g\d}\nu^{\a}x^{\b}_{;r}x^{\g}_{;j}x^{\d}_{;i}b^{rl}f_{;l}\\
            &=-h_{ri}A^{r}_{j}+\-R_{\a\b\g\d}\nu^{\a}\dot{x}^{\b}x^{\g}_{;i}x^{\d}_{;j}\\
            &=-h_{ri}A^{r}_{j}+\-R_{\a\b\g\d}x^{\a}_{;i}\nu^{\b}x^{\g}_{;j}\dot{x}^{\b}-\-R_{\a\b\g\d}x^{\a}_{;j}\nu^{\b}x^{\g}_{;i}\dot{x}^{\d},}
from which \eqref{Ev-h} follows.
}

Now we deduce evolution equations for several quantities mainly related to the desired Harnack estimates. We aim to deduce an evolution equation for the function
\eq{u=\fr{\dot{f}}{f}.}
 We begin with the evolution of the reparametrization field.

\Theo{lemma}{Ev-Tangent}{
Along the flow \eqref{Flow} there holds
\eq{\label{Ev-Tangent-1}\del_{t}\br{b^{km}f_{;m}}=fb^{ki}u_{;i}+ub^{ki}f_{;i}+A^{k}_{j}b^{jm}f_{;m}+\-R_{\a\b\g\d}\dot{x}^{\a}x^{\b}_{;i}\dot{x}^{\g}\nu^{\d}b^{ki}.}
}

\pf{
We calculate
\eq{u_{;i}&=\del_{t}\br{\fr{h_{ij}}{f}b^{jm} f_{;m}}\\
        &=\fr{h_{ij}}{f}\del_{t}\br{b^{jm}f_{;m}}-\fr{u}{f}f_{;i}-\fr{1}{f}h_{ir}A^{r}_{j}b^{jm}f_{;m}+\fr{1}{f}\-R_{\a\b\g\d}\dot{x}^{\a}x^{\b}_{;i}x^{\g}_{;j}\nu^{\d}b^{jm}f_{;m}\\
        &=\fr{h_{ij}}{f}\del_{t}\br{b^{jm}f_{;m}}-\fr{u}{f}f_{;i}-\fr{1}{f}h_{ir}A^{r}_{j}b^{jm}f_{;m}-\fr{1}{f}\-R_{\a\b\g\d}\dot{x}^{\a}x^{\b}_{;i}\dot{x}^{\g}\nu^{\d}}
and multiply by $fb^{ki}$ to obtain the result.
}

\Theo{lemma}{Ev-speed}{
Along the flow \eqref{Flow} there holds
\eq{\label{Ev-speed-1}\ddot{x}=u\dot{x}-fb^{ki}u_{;i}x_{;k}-\-R_{\a\b\g\d}\dot{x}^{\a}x^{\b}_{;i}\dot{x}^{\g}\nu^{\d}b^{ki}x_{;k}.}
}

\pf{We use \eqref{A2} and \eqref{Ev-Tangent-1} to deduce
\eq{\ddot{x}&=-\s \dot{f}\nu-\dot{x}_{;k}b^{ki}f_{;i}-x_{;k}\del_{t}\br{b^{ki}f_{;i}}\\
        &=-\s uf\nu+A^{l}_{k}b^{ki}f_{;i}x_{;l}-fb^{ki}u_{;i}x_{;k}-ub^{ki}f_{;i}x_{;k}\\
        &\hp{=}-A^{k}_{j}b^{ji}f_{;i}x_{;k}-\-R_{\a\b\g\d}\dot{x}^{\a}x^{\b}_{;i}\dot{x}^{\g}\nu^{\d}b^{ki}x_{;k}\\
        &=u\dot{x}-fb^{ki}u_{;i}x_{;k}-\-R_{\a\b\g\d}\dot{x}^{\a}x^{\b}_{;i}\dot{x}^{\g}\nu^{\d}b^{ki}x_{;k}.}
}

\Theo{lemma}{Ev-A}{
Along the flow \eqref{Flow} the tensor $(A^{i}_{j})$ satisfies
\eq{\dot{A}^{i}_{j}&=A^{k}_{j}A^{i}_{k}-\ip{x_{;r}}{\dot{x}}g^{ri}u_{;j}+uA^{i}_{j}+\br{fb^{kl}u_{;l}x^{\a}_{;k}}_{;j}g^{ri}\-g_{\a\b}x^{\b}_{;r}\\
                &\hp{=}+\br{\-R_{\e\b\g\d}\dot{x}^{\e}x^{\b}_{;l}\dot{x}^{\g}\nu^{\d}b^{kl}x^{\a}_{;k}}_{;j}g^{ri}\-g_{\a\mu}x^{\mu}_{;r}+\-R_{\a\b\g\d}x^{\a}_{;r}\dot{x}^{\b}x^{\g}_{;j}\dot{x}^{\d}g^{ri}.}
}

\pf{
Using \eqref{A2} and \eqref{Ev-speed-1} we obtain
\eq{\dot{A}^{k}_{i}x^{\a}_{;k}+A^{k}_{i}\dot{x}^{\a}_{;k}=-\fr{D}{dt}\br{\dot{x}^{\a}_{;i}}=-\br{\ddot{x}^{\a}}_{;i}+\-R^{\a}_{\b\g\d}\dot{x}^{\b}x_{i}^{\g}\dot{x}^{\d},}
so that
\eq{\dot{A}^{k}_{i}x^{\a}_{;k}&=A^{k}_{i}A^{l}_{k}x^{\a}_{;l}-u_{;i}\dot{x}^{\a}+uA^{k}_{i}x^{\a}_{;k}+\br{fb^{kl}u_{;l}x^{\a}_{;k}}_{;i}\\
                    &\hp{=}+\br{\-R_{\e\b\g\d}\dot{x}^{\e}x^{\b}_{;l}\dot{x}^{\g}\nu^{\d}b^{kl}x^{\a}_{;k}}_{;i}+\-R^{\a}_{\b\g\d}\dot{x}^{\b}x_{;i}^{\g}\dot{x}^{\d}.}
Multiplying this equation by $g^{rj}\-g_{\a\mu}x^{\mu}_{;r}$ and relabelling indices gives the result.
}

We need one more preliminary evolution equation.

\Theo{lemma}{Ev-B}{
The tensor $B_{ij}$ satisfies
\eq{\label{Ev-B-1}\dot{B}_{ij}&=u_{;ij}+T\ast\nabla u+\fr 1f \L_{ik}A^{k}_{j}+\fr 1f \-R_{\a\b\g\d}\dot{x}^{\a}x^{\b}_{;j}\dot{x}^{\g}\nu^{\d}_{;i}\\
            &\hp{=}+\fr 1f h^{s}_{i}\-g_{\a\mu}x^{\mu}_{;s}\br{\-R_{\e\b\g\d}\dot{x}^{\e}x^{\b}_{;l}\dot{x}^{\g}\nu^{\d}b^{rl}x^{\a}_{;r}}_{;j},}
where $T\ast\nabla u$ denotes the linear combination of arbitrary contractions of tensors with $\n u.$
}

\pf{
\eq{\dot{B}_{ij}&=-uB_{ij}+\fr 1f \dot{h}_{ik}A^{k}_{j}+\fr 1f h_{ik}\dot{A}^{k}_{j}\\
        &=-uB_{ij}-\fr 1f \br{fB_{ik}-\-R_{\a\b\g\d}\dot{x}^{\a}x^{\b}_{;i}x^{\g}_{;k}\nu^{\d}}A^{k}_{j} \\
        &\hp{=}+ B_{il}A^{l}_{j}+T\ast\n u+uB_{ij}+u_{;ij}+\fr 1f h_{ik}\-R_{\a\b\g\d}x^{\a}_{;r}\dot{x}^{\b}x^{\g}_{;j}\dot{x}^{\d}g^{rk}\\
        &\hp{=}+\fr 1f h_{ik}\br{\-R_{\e\b\g\d}\dot{x}^{\e}x^{\b}_{;l}\dot{x}^{\g}\nu^{\d}b^{sl}x^{\a}_{;s}}_{;j}g^{rk}\-g_{\a\mu}x^{\mu}_{;r},}
from which the result follows.
}


Now we have all ingredients to deduce the evolution equation of $u=\dot{f}/{f}.$ Also note that this will be the first place where the actual domain of definition of $f$ will play a role. Thus in the following Lemma we assume
\eq{f(s,\nu,h_{ij},g_{ij})=\p(s)F(\nu,h_{ij},g_{ij})}

\Theo{lemma}{Ev-u}{
Along the flow \eqref{Flow} there holds
\eq{\label{Ev-u-1}\dot{u}&=(\log\p)''\dot{s}^2+(\log\p)'\ddot{s}+\frac{\p'}{\p}\dot{s}f^{ij}\br{B_{ij}+\fr 1f \L_{ij}}\\
                &\hp{=}+2ff^{il}b^{jk}B_{kl}B_{ij}+ff^{ij,kl}\br{B_{kl}+\fr 1f\L_{kl}}\br{B_{ij}+\fr 1f\L_{ij}}\\
        &\hp{=}+f^{ij}u_{;ij}+T\ast\n u+\s\br{1-\fr{f^{ij}h_{ij}}{f}}\-R_{\a\b\g\d}\dot{x}^{\a}\nu^{\b}\dot{x}^{\g}\nu^{\d}\\
        &\hp{=}+\fr 2f f^{ij}h^{k}_{j}\-R_{\a\b\g\d}\dot{x}^{\a}x^{\b}_{;i}\dot{x}^{\g}x^{\d}_{;k}\\
        &\hp{=}+\fr 1f f^{ij}\-R_{\a\b\g\d;\e}x^{\a}_{;i}\dot{x}^{\b}x^{\g}_{;j}\dot{x}^{\d}\nu^{\e}-\fr 2f f^{ij}\-R_{\a\b\g\d;\e}x^{\a}_{;i}\dot{x}^{\b}x^{\g}_{;j}\nu^{\d}\dot{x}^{\e}.
        }}

\pf{
First of all we recall the formulas
\eq{\fr{\del f}{\del g_{ij}}=-\fr{\del f}{\del h_{mi}}h^{j}_{m},\quad \fr{\del^{2}f}{\del h_{ij}\del g_{kl}}=-g^{kj}f^{il}-f^{ij,km}h^{l}_{m}.}
Using the notation
\eq{f^{ij}=\fr{\del f}{\del h_{ij}},\quad f^{ij,kl}=\fr{\del^{2}f}{\del h_{ij}\del h_{kl} },}
 deduce from \eqref{Ev-g} and \eqref{Ev-h} that
\eq{u&=\fr 1f\br{\p'F\dot{s}+f^{ij}\dot{h}_{ij}-f^{mi}h^{j}_{m}\dot{g}_{ij}}\\
        &=\fr{\p'}{\p}\dot{s}+\fr 1f\br{-f^{ij}h_{ik}A^{k}_{j}+f^{ij}\-R_{\a\b\g\d}\dot{x}^{\a}x^{\b}_{;i}x^{\g}_{;j}\nu^{\d}+2f^{mj}h_{mk}A^{k}_{j}}\\
        &=(\log\p)'\dot{s}+f^{ij}\br{B_{ij}+\fr 1f\L_{ij}},}
where we have also used the symmetry of $f^{mj}h^{i}_{m}$.

 Thus
\eq{\label{dot u}\dot{u}&=(\log\p)''\dot{s}^2+(\log\p)'\ddot{s}+\frac{\p'}{\p}\dot{s}f^{ij}\br{B_{ij}+\fr 1f \L_{ij}}\\
            &\hp{=}+\br{g^{kj}f^{il}+f^{ij,km}h^{l}_{m}}\br{g_{kr}A^{r}_{l}+g_{lr}A^{r}_{k}}\br{B_{ij}+\fr 1f\L_{ij}}\\
        &\hp{=}-ff^{ij,kl}\br{B_{kl}-\fr 1f\L_{kl}}\br{B_{ij}+\fr 1f\L_{ij}}+f^{ij}\br{\dot{B}_{ij}+\del_{t}\br{\fr 1f\L_{ij}}}\\
        &=(\log\p)''\dot{s}^2+(\log\p)'\ddot{s}+\frac{\p'}{\p}\dot{s}f^{ij}\br{B_{ij}+\fr 1f \L_{ij}}\\
        &\hp{=}+\br{2ff^{il}b^{jk}B_{kl}+2ff^{ij,kl}B_{kl}}\br{B_{ij}+\fr 1f\L_{ij}}\\
        &\hp{=}-ff^{ij,kl}\br{B_{kl}-\fr 1f\L_{kl}}\br{B_{ij}+\fr 1f\L_{ij}}+f^{ij}\br{\dot{B}_{ij}+\del_{t}\br{\fr 1f{\L}_{ij}}}\\
        &=(\log\p)''\dot{s}^2+(\log\p)'\ddot{s}+\frac{\p'}{\p}\dot{s}f^{ij}\br{B_{ij}+\fr 1f \L_{ij}}\\
        &\hp{=}+\br{2ff^{il}b^{jk}B_{kl}+ff^{ij,kl}B_{kl}+f^{ij,kl}\L_{kl}}\br{B_{ij}+\fr 1f\L_{ij}}\\
        &\hp{=}+f^{ij}\br{\dot{B}_{ij}+\del_{t}\br{\fr 1f{\L}_{ij}}}\\
        &=(\log\p)''\dot{s}^2+(\log\p)'\ddot{s}+\frac{\p'}{\p}\dot{s}f^{ij}\br{B_{ij}+\fr 1f \L_{ij}}\\
        &\hp{=}+2ff^{il}b^{jk}B_{kl}B_{ij}+ff^{ij,kl}\br{B_{kl}+\fr 1f\L_{kl}}\br{B_{ij}+\fr 1f\L_{ij}}\\
        &\hp{=}+2f^{il}b^{jk}B_{kl}\L_{ij}+f^{ij}\br{\dot{B}_{ij}+\del_{t}\br{\fr 1f{\L}_{ij}}}.}
We use \eqref{Ev-B-1} to deduce
\eq{\label{Ev-u-3}&f^{ij}\br{\dot{B}_{ij}+\del_{t}\br{\fr 1f{\L}_{ij}}}\\
            =\ & f^{ij}u_{;ij}+T\ast\nabla u-\fr 1f f^{ij} \-R_{\a\b\g\d}\dot{x}^{\a}x^{\b}_{;i}\dot{x}^{\g}_{;j}\nu^{\d}\\
            \hp{=}&+\fr 1f f^{ij} h^{s}_{i}\-g_{\a\mu}x^{\mu}_{;s}\br{\-R_{\e\b\g\d}\dot{x}^{\e}x^{\b}_{;l}\dot{x}^{\g}\nu^{\d}b^{rl}x^{\a}_{;r}}_{;j}\\
            \hp{=}&+\fr 1f f^{ij} \-R_{\a\b\g\d}\dot{x}^{\a}x^{\b}_{;j}\dot{x}^{\g}\nu^{\d}_{;i}+f^{ij}\del_{t}\br{\fr 1f \-R_{\a\b\g\d}\dot{x}^{\a}x^{\b}_{;i}x^{\g}_{;j}\nu^{\d}}.}

Now we are facing the tedious exercise to balance as many curvature terms as possible. We begin by simplifying the most complicated ones. Among many others, the following terms are arising in \eqref{Ev-u-3}. The first term comes from differentiation $b^{rl}$ in the second line, the remaining ones from taking $\del_{t}$ of $1/f$ and $\dot{x}^{\a}$ in the last line. We get, using \eqref{Ev-speed-1} and the Codazzi equation,

\eq{\label{Ev-u-2}&-\fr 1f f^{ij}\-R_{\a\b\g\d}\dot{x}^{\a}x^{\b}_{;l}\dot{x}^{\g}\nu^{\d}h_{im;j}b^{ml}-\fr{u}{f}f^{ij}\L_{ij}+\fr 1f f^{ij}\-R_{\a\b\g\d}\ddot{x}^{\a}x^{\b}_{;i}x^{\g}_{;j}\nu^{\d}\\
    =& -\fr 1f f^{ij}\-R_{\a\b\g\d}\dot{x}^{\a}x^{\b}_{;l}\dot{x}^{\g}\nu^{\d}h_{ij;m}b^{ml}\\
    \hp{=}&-\fr 1f f^{ij}\-R_{\a\b\g\d}\dot{x}^{\a}x^{\b}_{;l}\dot{x}^{\g}\nu^{\d}b^{ml}\-R_{\mu\chi\rho\tau}\nu^{\mu}x^{\chi}_{;i}x^{\rho}_{;m}x^{\tau}_{;j}+T\ast\n u\\
    \hp{=}&-\fr 1f f^{ij}\-R_{\a\b\g\d}\-R_{\mu\chi\rho\tau}\dot{x}^{\mu}x^{\chi}_{;l}\dot{x}^{\rho}\nu^{\tau}b^{ml}x^{\a}_{m}x^{\b}_{;i}x^{\g}_{;j}\nu^{\d}\\
    =&\ \s\-R_{\a\b\g\d}\dot{x}^{\a}\nu^{\b}\dot{x}^{\g}\nu^{\d}+T\ast\n u.}
Note that to get the last identity, we used that the first curvature terms on the third and fourth lines appearing in \eqref{Ev-u-2} cancel each other out. Having this and \eqref{Ev-u-2} in mind, in the following calculation we collect the remaining ones in the order of appearance in \eqref{Ev-u-2}. We obtain
\eq{f^{ij}&\br{\dot{B}_{ij}+\del_{t}\br{\fr 1f{\L}_{ij}}}=f^{ij}u_{;ij}+T\ast\n u\\
                &\hp{=}+\s\-R_{\a\b\g\d}\dot{x}^{\a}\nu^{\b}\dot{x}^{\g}\nu^{\d}+\fr 1f f^{ij}\-R_{\a\b\g\d;\e}\dot{x}^{\a}x^{\b}_{;i}\dot{x}^{\g}\nu^{\d}x^{\e}_{;j}\\
                &\hp{=}+\fr 1f f^{ij}\-R_{\a\b\g\d}\dot{x}^{\a}_{;j}x^{\b}_{;i}\dot{x}^{\g}\nu^{\d}-\fr{\s}{f}f^{ij}h_{ij}\-R_{\a\b\g\d}\dot{x}^{\a}\nu^{\b}\dot{x}^{\g}\nu^{\d}\\
                &\hp{=}+\fr 2f f^{ij}\-R_{\a\b\g\d}\dot{x}^{\a}x^{\b}_{;i}\dot{x}^{\g}\nu^{\d}_{;j}+\fr 1f f^{ij}\-R_{\a\b\g\d;\e}\dot{x}^{\a}x^{\b}_{;i}x^{\g}_{;j}\nu^{\d}\dot{x}^{\e}\\
                &\hp{=}+\fr 1f f^{ij}\-R_{\a\b\g\d}\dot{x}^{\a}\dot{x}^{\b}_{;i}x^{\g}_{;j}\nu^{\d}+\fr 1f f^{ij}\-R_{\a\b\g\d}\dot{x}^{\a}x^{\b}_{;i}\dot{x}^{\g}_{;j}\nu^{\d}.}

 Recalling \eqref{dot u} we obtain
 \eq{\dot{u}&=(\log\p)''\dot{s}^2+(\log\p)'\ddot{s}+\frac{\p'}{\p}\dot{s}f^{ij}\br{B_{ij}+\fr 1f \L_{ij}}\\
             &\hp{=}+2ff^{il}b^{jk}B_{kl}B_{ij}+ff^{ij,kl}\br{B_{kl}+\fr 1f\L_{kl}}\br{B_{ij}+\fr 1f\L_{ij}}\\
         &\hp{=}-\fr 1f f^{ij}\-R_{\a\b\g\d}\dot{x}^{\a}x^{\b}_{;i}\dot{x}^{\g}_{;j}\nu^{\d}\\
         &\hp{=}+f^{ij}u_{;ij}+T\ast\n u+\s\br{1-\fr{f^{ij}h_{ij}}{f}}\-R_{\a\b\g\d}\dot{x}^{\a}\nu^{\b}\dot{x}^{\g}\nu^{\d}\\
        &\hp{=}+\fr 1f f^{ij}\-R_{\a\b\g\d}\dot{x}^{\a}_{;j}x^{\b}_{;i}\dot{x}^{\g}\nu^{\d}+\fr 1f f^{ij}\-R_{\a\b\g\d}\dot{x}^{\a}\dot{x}^{\b}_{;i}x^{\g}_{;j}\nu^{\d}\\
        &\hp{=}+\fr 1f f^{ij}\-R_{\a\b\g\d;\e}\dot{x}^{\a}x^{\b}_{;i}\dot{x}^{\g}\nu^{\d}x^{\e}_{;j}+\fr 1f f^{ij}\-R_{\a\b\g\d;\e}\dot{x}^{\a}x^{\b}_{;i}x^{\g}_{;j}\nu^{\d}\dot{x}^{\e}\\
        &\hp{=}+\fr 2f f^{ij}\-R_{\a\b\g\d}\dot{x}^{\a}x^{\b}_{;i}\dot{x}^{\g}\nu^{\d}_{;j}}
and due to the first Bianchi identity we arrive at
\eq{\dot{u}&=(\log\p)''\dot{s}^2+(\log\p)'\ddot{s}+\frac{\p'}{\p}\dot{s}f^{ij}\br{B_{ij}+\fr 1f \L_{ij}}\\
            &\hp{=}+2ff^{il}b^{jk}B_{kl}B_{ij}+ff^{ij,kl}\br{B_{kl}+\fr 1f\L_{kl}}\br{B_{ij}+\fr 1f\L_{ij}}\\
        &\hp{=}+f^{ij}u_{;ij}+T\ast\n u+\s\br{1-\fr{f^{ij}h_{ij}}{f}}\-R_{\a\b\g\d}\dot{x}^{\a}\nu^{\b}\dot{x}^{\g}\nu^{\d}\\
        &\hp{=}+\fr 2f f^{ij}h^{k}_{j}\-R_{\a\b\g\d}\dot{x}^{\a}x^{\b}_{;i}\dot{x}^{\g}x^{\d}_{;k}\\
        &\hp{=}+\fr 1f f^{ij}\-R_{\a\b\g\d;\e}\dot{x}^{\a}x^{\b}_{;i}\dot{x}^{\g}\nu^{\d}x^{\e}_{;j}+\fr 1f f^{ij}\-R_{\a\b\g\d;\e}\dot{x}^{\a}x^{\b}_{;i}x^{\g}_{;j}\nu^{\d}\dot{x}^{\e}.
        }
We may rewrite the last two terms using the second Bianchi identity as follows:
\eq{\bar{\nabla}_{x_{;j}}\bar{R}(\dot{x},\nu,\dot{x},x_{;i})&=-\bar{\nabla}_{\nu}\bar{R}(x_{;j},\dot{x},\dot{x},x_{;i})-\bar{\nabla}_{\dot{x}}\bar{R}(\nu,x_{;j},\dot{x},x_{;i})\\
&=\bar{\nabla}_{\nu}\bar{R}(x_{;j},\dot{x},x_{;i},\dot{x})+\bar{\nabla}_{\dot{x}}\bar{R}(\dot{x},x_{;i},x_{;j},\nu).
}
Therefore,
\eq{&\fr 1f f^{ij}\-R_{\a\b\g\d;\e}\dot{x}^{\a}x^{\b}_{;i}\dot{x}^{\g}\nu^{\d}x^{\e}_{;j}+\fr 1f f^{ij}\-R_{\a\b\g\d;\e}\dot{x}^{\a}x^{\b}_{;i}x^{\g}_{;j}\nu^{\d}\dot{x}^{\e}\\
&=\fr 1f f^{ij}\-R_{\a\b\g\d;\e}x^{\a}_{;i}\dot{x}^{\b}x^{\g}_{;j}\dot{x}^{\d}\nu^{\e}-\fr 2f f^{ij}\-R_{\a\b\g\d;\e}x^{\a}_{;i}\dot{x}^{\b}x^{\g}_{;j}\nu^{\d}\dot{x}^{\e}.
}
}

\section{Euclidean and Minkowski space}

In the case of vanishing ambient curvature, we obtain the following Harnack inequalities for anisotropic flows including dependence on the support function. They include the well-known inequalities from \cite{Andrews:09/1994} in the Euclidean space and they are completely new in the Minkowski space.

\Theo{thm}{R=0}{
Suppose $(\-R_{\a\b\g\d})=0$ and let $F=F(\nu,\mc{W})$ for fixed $\nu$ be $1$-homogeneous, positive and inverse concave. Let $0\neq p> -1$ and set 
\eq{f=\p(s)\mrm{sgn}(p)F^{p},} where $\p$ is a positive function of the support function $s$ satisfying
\eq{\s\p'\leq 0\quad\mbox{and}\quad \mrm{sgn}(p)\br{\fr{1-p}{p}\p'^2+\p''\p}\geq 0.}
Then along \eqref{Flow} there holds
\eq{\del_{t}\br{ft^{\fr{p}{p+1}}}\geq 0\quad\forall t>0.}
}

\pf{
Without ambient curvature \eqref{Ev-u-1} yields
\eq{\label{R=0-1}\dot{u}&=\br{\log\p}''\dot{s}^2+\br{\log\p}'\ddot{s}+\br{\log\p}'\dot{s}f^{ij}B_{ij}\\ &\hp{=}+f\br{2f^{il}b^{jk}+f^{ij,kl}}B_{ij}B_{kl}+f^{ij}u_{;ij}+T\ast\n u.}
The support function of a hypersurface is given by
\[s=\s\ip{x}{\nu},\]
hence
\eq{\dot{s}=\s\ip{\dot{x}}{\nu}=-\s f}
and
\eq{\ddot{s}&=-\s\dot{f}\\
            &=\fr{\p'}{\p}f^2-\s ff^{ij}B_{ij},}
where we used
\eq{\label{R=0-3} u=-\s \fr{\p'}{\p}f+f^{ij}B_{ij}.}
We obtain from \eqref{R=0-1} that
\eq{\dot{u}&=\fr{\p''}{\p}f^2-2\s\frac{\p'}{\p}ff^{ij}B_{ij}+f\br{2f^{il}b^{jk}+f^{ij,kl}}B_{ij}B_{kl}\\
            &\hp{=}+f^{ij}u_{;ij}+T\ast\n u.}

By the assumption on $F$, the curvature function
\[G=\mrm{sgn}(p)F^p\] 
satisfies
\[G^{ij,kl}+2G^{ik}b^{jl}\geq \fr{p+1}{p}G^{-1}G^{ij}G^{kl}\]
as bilinear forms on symmetric matrices, as \cite[p.~112]{Urbas:/1991} and a simple calculation reveals, and hence
\eq{\label{R=0-2}f\br{2f^{il}b^{jk}+f^{ij,kl}}B_{ij}B_{kl}=\p^2G\br{2G^{il}b^{jk}+G^{ij,kl}}B_{ij}B_{kl}.}

There holds
\eq{\del_{t}\br{ft^{\fr{p}{p+1}}}=ft^{\fr{p}{p+1}-1}\br{\fr{p}{p+1}+tu}.}
Set
\eq{w=tu+\fr{p}{p+1},}
then $w>0$ at $t=0$ if $p>0$ and $w<0$ at $t=0$ if $-1<p<0$. We aim to show that $w$ never changes sign. We first consider the case $p>0.$ There holds, using \eqref{R=0-2} and \eqref{R=0-3},
\eq{\dot{w}&=u+t\fr{\p''}{\p}f^2-2\s t\fr{\p'}{\p}ff^{ij}B_{ij}+tf\br{2f^{il}b^{jk}+f^{ij,kl}}B_{ij}B_{kl}\\
            &\hp{=}+f^{ij}w_{;ij}+T\ast\n w\\
            &\geq u+t\fr{\p''}{\p}f^2-2\s t\fr{\p'}{\p}ff^{ij}B_{ij}+t\fr{p+1}{p}\br{f^{ij}B_{ij}}^2\\
            &\hp{=}+f^{ij}w_{;ij}+T\ast\n w\\
            &=\fr{p+1}{p}uw-t\fr{p+1}{p}u^2+t\fr{\p''}{\p}f^2-2\s t\fr{\p'}{\p}fu\\
            &\hp{=}-2t\fr{\p'^2}{\p^2}f^2+t\fr{p+1}{p}u^2+2\s t\fr{p+1}{p}\fr{\p'}{\p}fu+t\fr{p+1}{p}\fr{\p'^2}{\p^2}f^2\\
            &\hp{=}+f^{ij}w_{;ij}+T\ast\n w\\
            &=\fr{p+1}{p}uw+\fr{2}{p}\s t\fr{\p'}{\p}fu+t\fr{1-p}{p}\fr{\p'^2}{\p^2}f^2+t\fr{\p''}{\p}f^2\\
            &\hp{=}+f^{ij}w_{;ij}+T\ast\n w\\
            &=\br{\fr{p+1}{p}u+\fr{2\s\p'}{p\p}f}w-\fr{2\s\p'}{(p+1)\p}f+\fr{t}{\p^2}f^2\br{\fr{1-p}{p}\p'^2+\p''\p}\\
            &\hp{=}+f^{ij}w_{;ij}+T\ast \nabla w.}
In case $p<0$ we have the same calculation with reversed inequality. Hence under the present assumptions $w$ never changes sign due to the maximum principle and the result follows.
}



% For example, if
%  \eq{\p(s)=s^{q},\quad q\leq0,}
%  then 
%  \eq{\p'=qs^{q-1},\quad \p''=q(q-1)s^{q-2},}
%  so that
%  \eq{\fr{1}{s^{2q-2}}\left((1-p)\br{\p'}^2+p\p''\p\right)=(1-p)q^2+pq(q-1)=q(q-p)\geq 0,}
%  whenever $q\leq0.$

% \textcolor[rgb]{1.00,0.00,0.00}{For another simple example, we may consider 
%  \eq{\p(s)=a+s^{q},\quad q<0,\quad a>0.}
% This choice of $\varphi$ breaks the homogeneity of the speed.  It is easy to check that for $0<p\leq 1$, the sum of the last two terms of (\ref{eq: extension of harnack}) is non-negative.}
\section{Spaces of constant curvature}
\Theo{lemma}{Spaceform}{
Let the ambient space $N$ be a spaceform of curvature $C$, i.e.,
\eq{\-R_{\a\b\g\d}=C\br{\-g_{\a\g}\-g_{\b\d}-\-g_{\a\d}\-g_{\b\g}}}
and let $f=\mrm{sgn}(p)F^p$, where $F$ is monotone and homogeneous of degree $1$. Then, if $p>0$, along \eqref{Flow} there holds
\eq{\label{Spaceform-1}\dot{u}&\geq\br{\fr 2p f^{ij}f^{kl}+ff^{ij,kl}}\br{B_{kl}+\fr 1f\L_{kl}}\br{B_{ij}+\fr 1f\L_{ij}}\\
        &\hp{=}-\fr{4C}{p}f^{ij}g_{ij}u+\fr{2C^2}{p}\br{f^{ij}g_{ij}}^2    \\
        &\hp{=}+f^{ij}u_{;ij}+T\ast\n u+\s\br{1-\fr{f^{ij}h_{ij}}{f}}\-R_{\a\b\g\d}\dot{x}^{\a}\nu^{\b}\dot{x}^{\g}\nu^{\d}\\
        &\hp{=}+2Cff^{ij}h_{ij}+\fr{2C}{f}f^{ij}h_{ij}b^r_lb^{ls}f_{;r}f_{;s}-\frac{2C}{f}f^{ij}b^s_if_{;j}f_{;s}}
and the reverse inequality holds in case $p<0.$
}

\pf{
For $1$-homogeneous and monotone curvature functions $F$ there holds
\eq{F^{ik}b^{jl}\eta_{ij}\eta_{kl}\geq F^{-1}\br{F^{ij}\eta_{ij}}^2}
for all symmetric matrices $\eta.$ Easy calculations yield
\eq{f^{ik}b^{jl}\eta_{ij}\eta_{kl}\geq \fr 1p f^{-1}\br{f^{ij}\eta_{ij}}^2.}
Note that in a spaceform $B$ is symmetric since
\eq{\L_{ij}=\-R_{\a\b\g\d}\dot{x}^{\a}x^{\b}_{;i}x^{\g}_{;j}\nu^{\d}=-Cg_{ij}\-g_{\a\d}\dot{x}^{\a}\nu^{\d}=Cfg_{ij}.}
Hence, if $p>0,$
\eq{2ff^{il}b^{jk}B_{ij}B_{kl}&\geq \fr 2pf^{ij}B_{ij}f^{kl}B_{kl}\\
                &=\fr 2p f^{ij}f^{kl}\br{B_{ij}+\fr 1f\L_{ij}}\br{B_{kl}+\fr 1f\L_{kl}}\\
                &\hp{=}-\fr 4p f^{-1}f^{ij}\L_{ij}u+\fr 2p f^{-2}f^{ij}\L_{ij}f^{kl}\L_{kl}.}
To obtain the last line of \eqref{Spaceform-1}, we insert \eqref{Flow} into $\-R_{\a\b\g\d}\dot{x}^{\a}x^{\b}_{;i}\dot{x}^{\g}x^{\d}_{;k}$ to obtain
\eq{\fr{2}{f}f^{ij}h^k_j\-R_{\a\b\g\d}\dot{x}^{\a}x^{\b}_{;i}\dot{x}^{\g}x^{\d}_{;k}&=2Cff^{ij}h_{ij}+\fr{2C}{f}f^{ij}h_{ij}b^r_lb^{ls}f_{;r}f_{;s}\\
        &\hp{=}-\frac{2C}{f}f^{ij}b^s_if_{;j}f_{;s}.}
The claim follows in case $p>0.$ The other case is similar.
}

\subsection*{The sphere}
If the ambient space is the sphere $\S^{n+1}$, we recover the class of speeds for which we could prove a Harnack inequality in \cite{BryanIvakiScheuer:12/2015}.

\Theo{cor}{Sphere}{
Let $N=\S^{n+1}$ and let $F$ be a monotone, convex and $1$-homogeneous curvature function. Let $0<p\leq 1$ and set $f=F^p,$ Then along \eqref{Flow} there holds
\eq{\del_t\br{ft^{\fr{p}{p+1}}}\geq 0.}
}

\pf{There holds
\eq{\fr 2p f^{ij}f^{kl}+ff^{ij,kl}=\fr{p+1}{p}f^{ij}f^{kl}+pfF^{p-1}F^{ij,kl}.}
Using \eqref{Spaceform-1} and
\[u=f^{ij}\left(B_{ij}+\frac{1}{f}\Lambda_{ij}\right),\quad f^{ij}\leq pfb^{ij},\]
we conclude that
$w=tu+\fr{p}{p+1}$
satisfies
\eq{\dot{w}\geq& u+t\fr{p+1}{p}u^2-\fr{4C}{p}f^{ij}g_{ij}w+f^{ij}w_{;ij}+T\ast\n w\\
&+\fr{4C}{p+1}f^{ij}g_{ij}+\fr{2tC^2}{p}\br{f^{ij}g_{ij}}^2+2tCpf^2\\
        =&\left(\fr{p+1}{p}u-\fr{4C}{p}f^{ij}g_{ij}\right)w+f^{ij}w_{;ij}+T\ast\n w\\
        &+\fr{4C}{p+1}f^{ij}g_{ij}+\fr{2tC^2}{p}\br{f^{ij}g_{ij}}^2+2tCpf^2.}
The result follows from the maximum principle.
}

By applying the dual flow method developed in \cite{Gerhardt:/2015} for flows in the sphere, we obtain some new reverse Harnack inequalities for expanding flows. For convenience, let us deduce the dual flow of a curvature flow
\eq{\label{normalFlow}\dot{x}=-f\nu}
in the sphere $\S^{n+1}$.
Denote by $\~x$ the normal embedded into $\R^{n+2},$ i.e.
\eq{\~x=y_{;\a}\nu^{\a}}
and differentiate to obtain the dual flow:
\eq{\label{Dualflow}\dot{\~x}&=y_{;\a\b}\dot{x}^{\b}\nu^{\a}+y_{;\a}\dot{\nu}^{\a}\\
                &=fy+y_{;\a}\dot{\nu}^{\a}\\
                &=fy+y_{;\a}g^{ij}f_{;i}x^{\a}_{;j}\\
                &=fy+g^{ij}\~h^k_j\~x_kf_{;i}.}
 But since $g_{ij}=\~h_{ik}\~h^k_j$, compare \cite[Thm.~9.2.5]{Gerhardt:/2006}, we obtain
 \eq{\dot{\~x}=fy+\~b^{ij}f_{;i}\~x_{;j}.}
 But
 \eq{f=f(\mc{W})=\fr{1}{f^{-1}(\~{\mc{W}}^{-1})}=\~f^{-1}(\~{\mc{W}})\equiv -\Phi(\~{\mc{W}}).}
 Hence the dual flow reads
 \eq{\label{DualFlow2}\dot{\~x}=-\Phi \~\nu-\~b^{ij}\Phi_{;i}\~x_{;j},}
 where $\~\nu$ is the pulled back version of $y$. We see that the dual flow is precisely of the form \eqref{Flow}. This is no surprise: From \eqref{normalFlow} and \eqref{Dualflow} we see that a flow has no tangential component precisely when its dual satisfies $\dot{\nu}=0.$
Thus for a class of inverse flow we obtain some reversed Harnack inequalities. 
 
 \Theo{cor}{Sphere}{
Let $N=\S^{n+1}$ and let $F$ be a monotone, inverse convex and $1$-homogeneous curvature function. Let $0<p\leq 1$ and set $f=-F^{-p}.$ Then along the expanding flow \eqref{normalFlow} there holds
\eq{\del_t\br{ft^{\fr{p}{p+1}}}\leq 0.}
In particular, this yields
\eq{\del_tf+\fr{p}{p+1}\fr{f}{t}\leq 0.}
}

\pf{
The dual \eqref{DualFlow2} with speed
\eq{\Phi(\~{\mc{W}})=-f(\mc{W})=F^{-p}(\mc{W})=\~F^p(\~{\mc{W}})}
satisfies the assumptions of \cref{Sphere} to deduce
\eq{\del_t\br{\Phi t^{\fr{p}{p+1}}}\geq 0.}
}

\subsection*{De Sitter space}

\Theo{cor}{DeSitter}{
Let $N$ be a Lorentzian spaceform with constant sectional curvature $C=1$. Let $F$ be a monotone, convex and $1$-homogeneous curvature function. Let  $p\geq 1$ and set $f=F^p$. Then along \eqref{Flow} there holds
\eq{\del_t\br{ft^{\fr{p}{p+1}}}\geq 0.}
}

\pf{The proof is the same as the proof of \cref{Sphere}. The assumption $p\geq 1$ is to ensure that the term in \eqref{Spaceform-1} involving the signature is non-negative.
}


\bibliographystyle{amsplain}
\bibliography{Bibliography.bib}

\end{document} 

\section{Mean curvature flow in Einstein manifolds}

\Theo{lemma}{Einstein}{
Let the ambient space $N$ be an Einstein manifold, i.e.
\eq{\-R_{\a\b}=\fr{R}{n+1}\-g_{\a\b},}
where $R$ is the constant scalar curvature of $\-g.$ Then for $f=H$ there holds
\eq{\label{Einstein-1}&\fr 1f f^{ij}\-R_{\a\b\g\d;\e}x^{\a}_{;i}\dot{x}^{\b}x^{\g}_{;j}\dot{x}^{\d}\nu^{\e}-\fr 2f f^{ij}\-R_{\a\b\g\d;\e}x^{\a}_{;i}\dot{x}^{\b}x^{\g}_{;j}\nu^{\d}\dot{x}^{\e}\\
=&-\fr 1f\-R_{\a\b\g\d;\e}\nu^{\a}\dot{x}^{\b}\nu^{\g}\dot{x}^{\d}\nu^{\e}.}
}

\pf{We plug 
\eq{g^{ij}x^{\a}_{;i}x^{\b}_{;j}=\-g^{\a\b}-\nu^{\a}\nu^{\b}}
into the left hand side of \eqref{Einstein-1} and use $\-R_{\a\b;\e}=0.$

}