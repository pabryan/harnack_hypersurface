\documentclass[10 pt]{amsart}

%\usepackage{etoolbox}
%\makeatletter
%\let\ams@starttoc\@starttoc
%\makeatother
%\makeatletter
%\let\@starttoc\ams@starttoc
%\patchcmd{\@starttoc}{\makeatletter}{\makeatletter\parskip\z@}{}{}
%\makeatother

%\usepackage[parfill]{parskip}

\usepackage[colorlinks=true,linkcolor=blue,citecolor=blue,urlcolor=blue]{hyperref}
\usepackage{bookmark}
\usepackage{amsthm,thmtools,amssymb,amsmath,amscd}

\usepackage{fancyhdr}
\usepackage{esint}
\bibliographystyle{/Users/J_Mac/Documents/Uni/TexTemplates/hamsplain}
\usepackage{enumerate}

\usepackage{pictexwd,dcpic}

\swapnumbers
\declaretheorem[name=Theorem,numberwithin=section]{thm}
\declaretheorem[name=Remark,style=remark,sibling=thm]{rem}
\declaretheorem[name=Lemma,sibling=thm]{lemma}
\declaretheorem[name=Proposition,sibling=thm]{prop}
\declaretheorem[name=Definition,style=definition,sibling=thm]{defn}
\declaretheorem[name=Corollary,sibling=thm]{cor}
\declaretheorem[name=Assumption,style=remark,sibling=thm]{ass}
\declaretheorem[name=Example,style=remark,sibling=thm]{example}
\declaretheorem[name=Notation,style=definition,sibling=thm]{notation}


\numberwithin{equation}{section}

\usepackage{cleveref}
\crefname{lemma}{Lemma}{Lemmata}
\crefname{prop}{Proposition}{Propositions}
\crefname{thm}{Theorem}{Theorems}
\crefname{cor}{Corollary}{Corollaries}
\crefname{defn}{Definition}{Definitions}
\crefname{example}{Example}{Examples}
\crefname{rem}{Remark}{Remarks}
\crefname{ass}{Assumption}{Assumptions}
\crefname{notation}{Notation}{Notation}



%Symbols
\renewcommand{\~}{\tilde}
\renewcommand{\-}{\bar}
\newcommand{\bs}{\backslash}
\newcommand{\cn}{\colon}
\newcommand{\sub}{\subset}

\newcommand{\N}{\mathbb{N}}
\newcommand{\Z}{\mathbb{Z}}
\newcommand{\Q}{\mathbb{Q}}
\newcommand{\R}{\mathbb{R}}
\newcommand{\C}{\mathbb{C}}
\renewcommand{\S}{\mathbb{S}}
\renewcommand{\H}{\mathbb{H}}
\newcommand{\K}{\mathbb{K}}
\newcommand{\Di}{\mathbb{D}}
\newcommand{\B}{\mathbb{B}}
\newcommand{\8}{\infty}

%Greek letters
\renewcommand{\a}{\alpha}
\renewcommand{\b}{\beta}
\newcommand{\g}{\gamma}
\renewcommand{\d}{\delta}
\newcommand{\e}{\epsilon}
\renewcommand{\k}{\kappa}
\renewcommand{\l}{\lambda}
\renewcommand{\o}{\omega}
\renewcommand{\t}{\theta}
\newcommand{\s}{\sigma}
\newcommand{\p}{\varphi}
\newcommand{\z}{\zeta}
\newcommand{\vt}{\vartheta}
\renewcommand{\O}{\Omega}
\newcommand{\D}{\Delta}
\newcommand{\G}{\Gamma}
\newcommand{\T}{\Theta}
\renewcommand{\L}{\Lambda}

%Mathcal Letters
\newcommand{\cL}{\mathcal{L}}
\newcommand{\cT}{\mathcal{T}}
\newcommand{\cA}{\mathcal{A}}
\newcommand{\cW}{\mathcal{W}}
\newcommand{\cH}{\mathcal{H}}
\newcommand{\cS}{\mathcal{S}}


%Mathematical operators
\newcommand{\INT}{\int_{\O}}
\newcommand{\DINT}{\int_{\d\O}}
\newcommand{\Int}{\int_{-\infty}^{\infty}}
\newcommand{\del}{\partial}
\newcommand{\de}[1]{\frac{\partial}{\partial #1}}
\newcommand{\n}{\nabla}
\newcommand{\II}[2]{\mrm{II}\br{#1,#2}}




\newcommand{\ip}[2]{\left\langle #1,#2 \right\rangle}
\newcommand{\fr}[2]{\frac{#1}{#2}}
\newcommand{\x}{\times}

\DeclareMathOperator{\dive}{div}
\DeclareMathOperator{\id}{id}
\DeclareMathOperator{\pr}{pr}
\DeclareMathOperator{\Diff}{Diff}
\DeclareMathOperator{\supp}{supp}
\DeclareMathOperator{\graph}{graph}
\DeclareMathOperator{\osc}{osc}
\DeclareMathOperator{\const}{const}
\DeclareMathOperator{\dist}{dist}
\DeclareMathOperator{\loc}{loc}
\DeclareMathOperator{\tr}{tr}
\DeclareMathOperator{\Rm}{Rm}
\DeclareMathOperator{\Rc}{Rc}
\DeclareMathOperator{\grad}{grad}


%Environments
\newcommand{\Theo}[3]{\begin{#1}\label{#2} #3 \end{#1}}
\newcommand{\pf}[1]{\begin{proof} #1 \end{proof}}
\newcommand{\eq}[1]{\begin{equation}\begin{alignedat}{2} #1 \end{alignedat}\end{equation}}
\newcommand{\IntEq}[4]{#1&#2#3	 &\quad &\text{in}~#4,}
\newcommand{\BEq}[4]{#1&#2#3	 &\quad &\text{on}~#4}
\newcommand{\br}[1]{\left(#1\right)}



%Logical symbols
\newcommand{\Ra}{\Rightarrow}
\newcommand{\ra}{\rightarrow}
\newcommand{\hra}{\hookrightarrow}
\newcommand{\mt}{\mapsto}

%Fonts
\newcommand{\mc}{\mathcal}
\renewcommand{\it}{\textit}
\newcommand{\mrm}{\mathrm}

%Spacing
\newcommand{\hp}{\hphantom}


\parindent 0 pt

%\protected\def\ignorethis#1\endignorethis{}
%\let\endignorethis\relax
%\def\TOCstop{\addtocontents{toc}{\ignorethis}}
%\def\TOCstart{\addtocontents{toc}{\endignorethis}}











%\usepackage[inline]{showlabels}
\usepackage{mathrsfs}
\DeclareMathOperator{\ad}{ad}

\begin{document}

\title[Harnack inequalities for curvature flows]{Harnack inequalities for curvature flows in Riemannian and Lorentzian manifolds}
\maketitle
\section{Introduction}
\label{sec:intro}
Let $N=N^{n+1}$ be a Riemannian or Lorentzian manifold and let $M=M^{n}$ be a smooth and orientable manifold. Let $\s = 1$ in the Riemannian case and $\s = -1$ in the Lorentzian case. Let
\eq{x\cn M\x[0,T^{\ast})\ra N}
be a family of strictly convex spacelike embeddings (the term ``spacelike'' in the Riemannian case should be ignored), which evolve by the curvature flow
\eq{\label{Flow}\dot{x}=-\s f\nu-b^{ij}f_{;i}x_{;j},}
where $\nu$ is a unit normal vector field along $M_{t}=x(M,t)$ (which satisfies $\s=\ip{\nu}{\nu}$ from the spacelike condition), $(b^{ij})$ is the inverse of the second fundamental form $(h_{ij}),$ indices appearing after a semicolon denote the components of the covariant derivative with respect to the induced metric $(g_{ij})$. The speed, $f$ is a smooth, increasing function of the principal curvatures.

The flow \eqref{Flow} is equivalent to the curvature flow
\eq{\label{FlowStandard}
\bar{x}&:M\times [0,T^{\ast})\to N\\
\dot{\bar{x}} &= -\s \bar{f} \bar{\nu}.}
by letting \(r_t:M\to M\) to be the flow of the vector field $b^{ij}f_{;i}$ on $M$ and  $\bar{x} (\xi, t) := x (r_t(\xi), t)$. We call $\bar{x}$  the standard parameterization as in \cite{Andrews:09/1994}.

Harnack inequalities are point-wise derivative estimates which usually enable one to compare a solution to a curvature flow at different points in space-time.
Central to our approach in obtaining Harnack inequalities for a class of curvature flows (\ref{FlowStandard}) is a reparameterization of the flow given by the flow \eqref{Flow}.
In a Euclidean background, $N = \R^{n+1}$, one may consider the unit normal at time $t$, as the Gauss map $\nu_t : M \to \S^n$, which is a diffeomorphism whenever $x(M,t)$ is strictly convex. The Gauss map parameterization $y_t: \S^n \to \R^{n+1}$ \cite{Andrews:09/1994} is such that $\nu_t(y_t(z)) = z$ for all $z \in \S^n$ whence $\dot{\nu} = 0$. Furthermore, calculations may be performed with respect to the fixed, canonical, round metric $g_{\operatorname{can}}$. These two properties, a static metric and static normal provide immense benefit, not only in simplifying the generally long computations associated with differential Harnack inequalities, but also by lending insight into why such long computations yield such a simple, elegant differential Harnack inequality.

The Gauss map parameterization just described is manifestly Euclidean, and given the utility of such a parameterization, analogous results in other background spaces should be highly prized. The cornerstone of our approach is that the normal \(\nu\) is static in the parameterization \eqref{Flow} and the time derivative of the induced metric \(g\) is only felt through the changing parameterization, $x$. See (\ref{BE-n}), analogous to the Gauss map parameterization, valid in arbitrary backgrounds.

Another crucial aspect of our approach is that it also addresses the question of how to determine the appropriate Harnack quantity. The philosophy put forward by Hamilton in \cite{Hamilton:/1993,Hamilton:/1995} is that equality should be attained on expanding solitons, just as equality in the Li-Yau Harnack inequality \cite{LiYau:/1986} is attained by the heat kernel, itself an expanding soliton. Thus Hamilton follows a procedure of differentiating the soliton equation to obtain soliton identities which eventually lead to the appropriate form for the Harnack quantity. In \cite{Andrews:09/1994}, Andrews showed that under the Gauss map parameterization, one may consider the evolution of the support function to which the Li-Yau approach may be applied to determine the Harnack quantity. This is also effectively our approach, although for general backgrounds we do not have the luxury of a suitable support function, but have the great fortune of the reparameterization \eqref{Flow}. Namely, we define
\begin{equation}
\label{eq:Q}
u = \frac{\dot{f}}{f}
\end{equation}
for the Harnack quantity. For shrinking flows (i.e., positive $f$) we have $u = \partial_t \ln f$ just as for Li-Yau \cite{LiYau:/1986} and \cite{Andrews:09/1994}. For expanding flows (i.e., negative $f$), we have the analogous formulation, $u =\partial_t \ln (-f)$, taking into account the sign, and it seems simpler to use the form given in \eqref{eq:Q} for both cases. Let us define $\bar{f}(\xi,t):=f(r_t(\xi), t).$ Then writing $u$ in the standard parameterization, we find that
\eq{
  \frac{\dot{f}}{f} &= \fr{\partial_t \bar{f} - df (\partial_t r_t)}{f} \\
 &= \frac{\partial_t\bar{f}-\bar{b}(\nabla \bar{f},\nabla \bar{f})}{\bar{f}}
}
which is precisely the standard Harnack quantity in the Euclidean space.


Our first theorem includes previously known Harnack inequalities in the Euclidean space and extends them by allowing the speed dependence on the ``support function". Furthermore, it provides Harnack inequalities for a class of curvature flows in the Minkowski space which are completely new.

Suppose there is a subgroup $G$ of future preserving isometries of the Minkowski space  such that $I(x(M))=x(M)$ for all $I\in G$ and $G$ acts properly discontinuously on $M.$ Let us put $K=M/G$. If $K$ is compact, we say that $M$ is co-compact. Let $I_{\ast}$ denote the linear part of $I\in G$ (e.q., $I=I_{\ast}+\vec{v}$ such that $I_{\ast}\in O^+(n,1),~\vec{v}\in\mathbb{R}^{n,1}$,  where $O^+(n,1)$ is the space of future-preserving linear transformations \emph{preserving the Lorentzian inner product} and $\mathbb{R}^{n,1}$ denotes the Minkowski space) and also put $G_{\ast}=\{I_{\ast}:I\in G\}.$ If in addition, $G_{\ast}=G$, we say $M$ is standard.

Write $\mathbb{H}^n$ for the hyperbolic space. A function $\psi:\mathbb{H}^n\to \mathbb{R}$ is called $G_{\ast}$-invariant, if $\psi(I_{\ast}x)=\psi(x)$ for all $x\in \mathbb{H}^n$ and $I\in G.$ Therefore, $\psi:\mathbb{H}^n/G_{\ast}\to \mathbb{R}$ is well-defined.
\Theo{thm}{Euclidean}{
Let $s=\sigma\langle x,\nu\rangle$ be the support function and assume $F$ is a monotone, inverse concave, $1$-homogeneous curvature function. Let $0\neq p> -1$ and set
$f=\p(s)\psi(\nu)\mrm{sgn}(p)F^{p},$ where $\p$ is a positive smooth function of $s$ satisfying
\eq{\s\p'\leq 0\quad\mbox{and}\quad \mrm{sgn}(p)\br{\fr{1-p}{p}\p'^2+\p''\p}\geq 0.}
Suppose one of the following conditions holds
\begin{enumerate}
  \item $N=\mathbb{R}^{n+1}$, $\psi$ is a positive smooth function defined on the unit sphere and the solution is strictly convex and if $\varphi\neq 1$ then $s(\cdot,t)>0$ for all $t.$
  \item $N=\mathbb{R}^{n,1}$  (Minkowski space), $\varphi=\psi\equiv1$,  the solution is co-compact, spacelike and strictly convex.
  \item $N=\mathbb{R}^{n,1}$,  $\varphi\equiv1$, the solution is co-compact, spacelike and strictly convex and $\psi:\mathbb{H}^n\to \mathbb{R}_+$ is a $G_{\ast}$-invariant, smooth function.
  \item $N=\mathbb{R}^{n,1}$, the solution is standard, spacelike, strictly convex, and  $\psi:\mathbb{H}^n\to \mathbb{R}_+$ is a $G_{\ast}$-invariant, smooth function and $s(\cdot,t)>0$ for all $t.$
\end{enumerate}
Then the following Harnack inequality holds in the standard parametrization:
\eq{\del_{t}f-b(\nabla f,\nabla f)+\frac{p}{(p+1)t}f\geq 0,\quad \forall t>0.}
}
\Theo{rem}{s well-defined on K}{Note for a standard, spacelike and strictly convex hypersurface, $s$ is well-defined on $K:$
\begin{align*}
s(Ix)&=-\langle Ix,\nu(Ix)\rangle\\
&=-\langle Ix,I_{\ast}\nu(x)\rangle\\
&=-\langle Ix,I\nu(x)\rangle\\
&=-\langle x,\nu(x)\rangle=s(x).
\end{align*}
}
\Theo{rem}{}{Let $\mathcal{F}$ denote the interior of $\{\langle x,x\rangle\leq 0, x_0\geq 0\}$. If $M$ is a standard, spacelike, strictly convex hypersurface that is contained in $\mathcal{F}$, then $M$ has a positive support function.}
\Theo{rem}{}{Consider $x(M,t)=\sqrt{2F(1,\cdots,1)t}\mathbb{H}^{n}$ for $t\in(0,\infty)$, a solution to the expanding flow with a positive, 1-homogeneous speed $f=F$ (assuming $\varphi=\psi\equiv1$) and $N=\mathbb{R}^{n,1}$. Then equality holds in the Harnack inequality. Note that if $t\to 0$, then $x(M,t)\to\{\langle x,x\rangle=0: x_0\geq 0\}$ (e.q., boundary of $\mathcal{F}$) with support function  equal to zero.}
In \cite{ChowHamilton:/2004}, Chow and Hamilton introduce an interesting fully nonlinear heat flow for negatively (or positively) curved metrics on a 3-manifold, called the ``cross curvature flow'' (in short ``XCF"). This nonlinear curvature flow of metrics is dual to the Ricci flow in the following sense. The identity map from a Riemannian 3-manifold to itself, where the domain manifold has the cross curvature tensor as the metric (assuming the sectional curvature is either everywhere negative or everywhere positive), is harmonic, while the identity map from a Riemannian 3-manifold to itself, where the target manifold has the  Ricci curvature tensor as the metric (assuming the Ricci curvature is either everywhere negative or everywhere positive), is harmonic. Chow and Hamilton prove a monotonicity formula for XCF and give strong indications that the XCF should deform any negatively curved metric on a compact 3-manifold to a hyperbolic metric, modulo scaling. Also, they express strong hopes that the XCF should enjoy a Harnack inequality. Recently, it is appeared in \cite{AndrewsChenFangMcCoy:/2015} that if the universal cover of the initial 3-manifold is isometrically embeddable as a hypersurface in Minkowski 4-space (or Euclidean 4-space), then the Gauss curvature flow of the hypersurface yields the cross curvature flow of the induced metric. When, also, the manifold is closed, the global existence and convergence hold \cite{AndrewsChenFangMcCoy:/2015}. In that case, it is a corollary of Theorem \ref{Euclidean} that indeed a Harnack estimate for XCF exists; see inequality (\ref{cross harnack}).

In ambient space of non constant curvature we obtain several new results. We start with the most general ambient space we can, up to our current state of knowledge, allow for. Namely we obtain a Harnack inequality with bonus term for the mean curvature flow in locally symmetric Riemannian Einstein manifolds of non-negative sectional curvature:
\Theo{thm}{Einstein}{
Let $N$ be a locally symmetric Riemannian Einstein manifold of non-negative sectional curvature. Then under the mean curvature flow there holds
\[\del_{t}H-b(\nabla H,\nabla H)-\fr{\-R}{n+1}H+\fr{1}{2t}\geq 0,\]
where $\-R$ is the constant scalar curvature of $N$.
}

The next theorem includes our Harnack inequalities from \cite{{BryanIvakiScheuer:12/2015}} and presents new Harnack inequalities in de Sitter space.
\Theo{thm}{Harnack Inequality}{
Let $F$ be a monotone, convex, $1$-homogeneous curvature function and $f =F^p$ with $0< p\leq 1$.
Suppose either
\begin{enumerate}
  \item $N$ is the sphere and the solution is strictly convex or
  \item $N$ is the de sitter and the solution satisfies $0<\kappa_i\leq 1.$
\end{enumerate}
Then the following Harnack inequality holds in the standard parametrization:
\[
\partial_t F^p-b(\nabla F^p,\nabla F^p)+\frac{p}{(p+1)t}F^p\geq 0,\quad \forall t>0.
\]
}
Furthermore, employing  duality, we obtain ``pseudo"-Harnack inequalities for a class of curvature flows in the spherical and the hyperbolic space.
\Theo{thm}{pseudo Harnack Inequality}{
Let $F$ be a monotone, inverse convex, $1$-homogeneous curvature function and $f =-F^p$ with $-1\leq p<0$ .
Suppose either
\begin{enumerate}
  \item $N$ is the sphere and the solution is strictly convex or
  \item $N$ is the hyperbolic space and the solution is horoconvex.
\end{enumerate}
Then the following inequality holds in the standard parametrization:
\[
\partial_t F^p+\frac{p}{(p-1)t}F^p\geq 0,\quad \forall t>0.
\]
}
The term pseudo-Harnack reflects the fact that the inequality in \Cref{pseudo Harnack Inequality} does not have the gradient term as apposed to the inequalities in Theorems \ref{Euclidean} and \ref{Harnack Inequality} and thus would not allow one to compare the solution at different points in space-time, nevertheless, it is a point-wise estimate on $\dot{F}$, which is independent of the initial data. This new type of inequality suggests while in a negatively curved ambient space the standard Harnack quantity $u$ may fail to yield any interesting inequality, yet a weaker form (obtained by dropping the gradient term) may provide a useful inequality.
\section{Background and Notation}
%\label{sec:background}

\subsection{Notation and Basic Definitions}
\label{subsec:bg_notation}

For a semi-Riemannian manifold $(M,g)$ we stipulate the following convention for the flat- and sharp-operators. For a tensor field
\eq{T\cn \mc{T}^{k}M\ra \mc{T}M}
let
\eq{T^{\flat}(X_{1},\dots, X_{k+1})=g(T(X_{1},\dots,X_{k}),X_{k+1})} and for a pure covariant tensor the sharp operator is defined by the requirement
\eq{\br{S^{\sharp}}^{\flat}=S;}
equivalently,
\[S(X_{1},\dots, X_{k+1})=g(S^{\sharp}(X_{1},\dots, X_{k}),X_{k+1}).\]
If the metric is denoted by some other symbol, i.e., $\-g$, these operator will also be furnished like this, e.g., $\-\flat$.

For an embedding into a semi-Riemannian manifold $(N^{n+1},\-g)$,
\eq{x\cn M^{n}\ra N^{n+1},}
we let $g=x^{*}\-g$ be the induced metric and the second fundamental form is defined by the Gaussian formula
\eq{\-\n_{x_{*}X}(x_{*}Y)=x_{*}\br{\n_{X}Y}-\s h(X,Y)\nu\quad\forall X,Y\in \mc{T}M, }
where $\-\n$ and $\n$ denote the Levi-Civita connections of $\-g$ and $g$ respectively. The Weingarten map is given by
\eq{g(\mc{W}(X),Y)=h(X,Y).}
From this, differentiating $0=\-g(\nu,{x_{*}Y})$, we obtain the Weingarten equation
\eq{\label{Weingarten}\-g(\-\n_{x_{*}X}\nu,x_{*}Y)=h(X,Y)\quad\forall X,Y\in \mc{T}M.}
Generally, geometric quantities of the ambient manifold are denoted by an overbar, e.g., our definition of the $(1,3)$ Riemannian curvature tensor of $\-g$ is given by
\eq{\overline{\mrm{Rm}}(\-X,\-Y)\-Z=\-\n_{\-X}\-\n_{\-Y}\-Z-\-\n_{\-Y}\-\n_{\-X}\-Z-\-\n_{[\-X,\-Y]}\-Z}
and the $(0,4)$ version is
\eq{\overline{\mrm{Rm}}^{\-\flat}(\-X,\-Y,\-Z,\-W)=\-g\br{\overline{\mrm{Rm}}(\-X,\-Y)\-Z,\-W},}
where we suppress the $\-\flat$, if no ambiguities are possible.
Hence we have the Codazzi equation
\eq{\label{Codazzi}\br{\n_{Z}h}(X,Y)=\n h (X,Y,Z)=\n h(X,Z,Y)-\overline{\mrm{Rm}}(\nu,X,Y,Z).}
Note that
\[\nabla h (Z,X,Y)=g(\nabla_Y \mc{W}(X),Z).\]
Therefore, we may rewrite (\ref{Codazzi}) equivalently as follows
\eq{\n_Y\mc{W}(X)=\n_X\mc{W}(Y)-\br{\overline{\mrm{Rm}}(X,Y)\nu}^{\top},}
where we stipulate that whenever we insert $X\in \mc{T}M$ into ambient tensors, we understand $X$ to be the push-forward $x_{*}X.$

For a bilinear form $B$, $B^t$ denotes its transpose,
\eq{B^t(X,Y)=B(Y,X)}
and $B_{\mrm{sym}}$ denotes its symmetrization, i.e.,
\eq{B_{\mrm{sym}}=\fr 12(B+B^t).}

Let $S$ and $T$ be two $2$- tensors and $S\otimes T$ be their tensor product. If there is only one way to form a new $2$- tensor by tracing $S\otimes T$, we denote it by $S\ast T$. For example if $T\in T^{0,2}(M)$ and $S\in T^{2,0}(M)$ are both symmetric, then $S\ast T\in T^{1,1}(M)$ is uniquely defined. Similarly, if $T\in T^{2,0}(M)$ is symmetric and $S\in T^{1,1}(M)$, then there is only one way to form a traced tensor $T\ast S\in T^{2,0}(M)$.

% For a tensor $T$, $T_{;ij}$ denotes the covariant derivative of $T$, first with respect $x_{;i}$ and then with respect to $x_{;j}.$ The normal $\nu$ is supposed to be the same normal as in the Gaussian formula
% \eq{x^{\a}_{;ij}:=\fr{\del^2 x^{\a}}{\del \xi^i\del \xi^j}+\-\G^{\a}_{\b\g}\fr{\del x^{\b}}{\del \xi^i}\fr{\del x^{\g}}{\del \xi^j}-\G^k_{ij}\fr{\del x^{\a}}{\del \xi^k}=-\s h_{ij}\nu^{\a}.}
% Recall also the Weingarten equation,
% \eq{\nu_{;i}=h^{k}_{i}x_{;k}.}
% Geometric quantities of the ambient space are denoted with an over-bar, e.g., $(\-g_{\a\b})$ for the metric and Greek indices range from $0$ to $n.$ Induced quantities are denoted for example by $(g_{ij})$, $(h_{ij}),$ where Latin indices range from $1$ to $n.$
% Our definition of the $(1,3)$ Riemannian curvature tensor is given by
% \eq{\mrm{Rm}(X,Y)Z=\n_{Y}\n_{X}Z-\n_{X}\n_{Y}Z-\n_{[Y,X]}Z}
% and the $(0,4)$ version is
% \eq{\mrm{Rm}(W,X,Y,Z)=\ip{W}{\mrm{Rm}(X,Y)Z}.}
% Hence the curvature tensor of the ambient space reads in coordinates
% \eq{\-R^{\a}_{\b\g\d}= [\-R(\partial_{\delta}, \partial_{\gamma})\partial_{\beta}]^{\alpha} = \-\G^{\a}_{\b\d,\g}-\-\G^{\a}_{\b\g,\d}+\-\G^{\e}_{\b\d}\-\G^{\a}_{\e\g}-\-\G^{\e}_{\b\g}\-\G^{\a}_{\e\d},}
% where $\-\G$ are the Christoffel symbols of the metric $\-g$ and a comma denotes ordinary differentiation.
% We obtain the Ricci identities by differentiating a contravariant vector field $\eta=\eta^{\a}\del_{\a}\cn$
% \eq{\eta^{\a}_{;\b\g}&=\br{\eta^{\a}_{;\b}}_{,\g}+\-\G^{\a}_{\d\g}\eta^{\d}_{;\b}-\-\G^{\d}_{\b\g}\eta^{\a}_{;\d}\\
%                     &=\eta^{\a}_{,\b\g}+\-\G^{\a}_{\b\d,\g}\eta^{\d}+\-\G^{\a}_{\b\d}\eta^{\d}_{,\g}+\-\G^{\a}_{\d\g}\eta^{\d}_{,\b}+\-\G^{\a}_{\d\g}\-\G^{\d}_{\e\b}\eta^{\e}-\-\G^{\d}_{\b\g}\eta^{\a}_{,\d}-\-\G^{\d}_{\b\g}\-\G^{\a}_{\e\d}\eta^{\e}.}
% In Riemannian normal coordinates:
% \eq{\eta^{\a}_{;\b\g}=\eta^{\a}_{,\b\g}+\-\G^{\a}_{\b\d,
% \g}\eta^{\d}&=\eta^{\a}_{,\g\b}+\-\G^{\a}_{\g\d,\b}\eta^{\d}+\br{\-\G^{\a}_{\b\d,\g}-\-\G^{\a}_{\g\d,\b}}\eta^{\d}\\
%             &=\eta^{\a}_{;\g\b}+\-R^{\a}_{\d\g\b}\eta^{\d}=\eta^{\a}_{;\g\b}-\-R^{\a}_{\d\b\g}\eta^{\d}.}
% There also holds
% \eq{\-R_{\a\b\g\d}=\-g_{\a\e}\-R^{\e}_{\b\g\d}.}
% Thus the Ricci tensor is given by
% \eq{\-R_{\a\b}=\-R^{\g}_{\a\g\b}.}


% \subsection{The canonical space-time connection}

% It turns out that the calculation of the evolution equations for \eqref{Flow} simplify a bit when we use the canonical space-time connection for time dependent families of embeddings, as described very nicely in \cite[Sec.~6.3]{AndrewsHopper:/2011}. Let us quickly recall the main features of this connection.

% Given a smooth time-dependent family of metrics $g(t)$ on a manifold $M$, we can consider the {\it{spatial tangent bundle}} on $\R\x M$ given by
% \eq{\mathscr{S}=\{V\in T(\R\x M)\cn dt(V)=0\},}
% which gives the product decomposition
% \eq{T(\R\x M)=\R\del_{t}\oplus \mathscr{S}.}
% A unique connection
% \eq{\n\cn TM\x \mathscr{S}\ra \mathscr{S}}
%  on the spatial tangent bundle can be defined such that
% \begin{itemize}
% \item[(i)] $\n$ is compatible with $g$,
% \item[(ii)] $\n$ is spatially symmetric, i.e., for $X,Y\in \mc{T}(\mathscr{S})$ we have
% \eq{\n_{X}Y-\n_{Y}X=[X,Y]}
% \item[(iii)] $\n$ is irrotational, i.e., the tensor $S\in \mc{T}(\mathscr{S}^{*}\otimes \mathscr{S})$, defined by
% \eq{S(V)=\n_{\del_{t}}V-[\del_{t},V]}
% is self-adjoint with respect to $g,$
% \end{itemize}
% compare \cite[Thm.~6.9]{AndrewsHopper:/2011}. Note also that
% \eq{\n_{\del_{t}}\del_{i}={\G_{0i}}^{k}\del_{k},}
% where
% \eq{{\G_{0i}}^{k}=\fr 12 g^{kj}\dot{g}_{ij}}
% are the remaining Christoffel symbols (besides the one from the spatial Levi-Civita connections).
% With the definition
% \eq{\n\del_{t}=0,}
% this connection can be extended to a linear connection
% \eq{\n\cn T(\R\x M)\x T(\R\x M)\ra T(\R\x M).}
% However note that this connection is not symmetric. Still, since $\n$ is a linear connection, covariant differentiation of tensors and function is now well defined such that it obeys the product rule and commutes with traces.


\subsection{Speed Functions}
\label{subsec:bg_speed}
\subsection*{Curvature functions}
We collect some well-known facts about curvature functions. A thorough treatment can be found in \cite[Ch.~2]{Gerhardt:/2006}.

Let $M$  be a smooth manifold. A curvature function is a smooth map
\eq{F\cn T^{1,1}(M)\ra \R,}
such that $\fr{\del F}{\del x^i}=0,$ when representing elements in $T^{1,1}(M)$ locally  by coordinates $(x^i,h^k_l)$. Given a curvature function $F$ and a metric $g$, on $T^{0,2}(M)$ one can define
\[\~F(h):=F\br{g^{-1}\ast \fr 12 (h+h^t)}\equiv F(\cW). \]
Hence we have \footnote{
Let $h=h(s)$ be an arbitrary curve in $T^{0,2}_p(M)$ such that $h'(0)$ is symmetric for a fixed $p\in M$. Then
\[\fr{d}{ds} \~F(g,h)_{|s=0}=\fr{\del \~F}{\del h}h'(0)=\fr{\del F}{\del \cW}\cW'(0)=\fr{\del F}{\del \cW}\br{g^{-1}\ast h'(0)}=\br{\fr{\del F}{\del \cW}\ast g^{-1}}h'(0).\]}
\eq{d_h \~F(\eta):=\fr{\del \~F}{\del h}(\eta)=\fr{\del F}{\del \cW}\ast g^{-1}(\eta)=dF\ast g^{-1}(\eta)}
for all symmetric bilinear forms $\eta$.
For $T\in T^{1,1}(M)$, we have\footnote{$F^i_j T^j_i=F^{ik}g_{kj}T^j_i=F^{ik}\br{g(T(\cdot),\cdot)}_{ik}$.}
\eq{dF(T)=\operatorname{Tr}(dF\circ T)=\operatorname{Tr}((d_h\~F\ast g) \circ T)=\operatorname{Tr}_{d_h\~F}g(T(\cdot),\cdot).}
We will use the following terminology regarding curvature functions.

\Theo{defn}{CF}{~

\begin{description}
  \item[i] For a metric $g\in T^{0,2}(M)$, $T^{1,1}_g(M)$ denotes the subbundle of $T^{1,1}(M)$ such that the fibre at $p\in M$ is the vector space of $g(p)$-self-adjoint endomorphisms.
  \item[ii] A curvature function $F$ on $M$ is called {\it{(strictly) monotone}}, if for all metrics $g$ and for all $\cW\in T^{1,1}_g(M)$ the bilinear form 
\eq{dF_{\cW}\ast g^{-1} }
is non-negative (positive) definite.
  \item[iii] $F$ is called {\it{homogeneous of degree 1}}, if for all metrics $g$ there holds
\eq{F(\l \cW)=\l F(\cW)\quad \forall \cW\in T^{1,1}_g(M)~\forall \l>0.}
  \item[iv] $F$ is called {\it{concave (convex)}}, if for all metrics $g$ the restriction of $F$ to $T^{1,1}_{g}(M)$ is concave (convex) at each $p\in M.$
\end{description}
}


From now on, we do not distinguish $F$ and $\~F$ by different notation and write $F$ for both functions. This will not cause ambiguities, since it will always be clear from the notation $d_h F$, if we actually mean $\~F.$




A curvature function $F$ is called {\it{inverse concave}} if
\[\~F(\mc{W}):=\fr{1}{F(\mc{W}^{-1})}\]
is concave, where we suppressed potential dependence on other arguments like $s$ and $\nu.$

The following lemma about inverse concave curvature functions is basically known from \cite[p.~112]{Urbas:/1991}, but since there it appears in a different form, let us deduce it again in our notation.

\Theo{lemma}{CF}{
Let $F$ be a monotone, $1$-homogeneous and inverse concave curvature function. Then for all $p\neq 0$ the curvature function $f=\mrm{sgn}(p)F^p$
satisfies

\eq{d^2f(\eta,\eta)+2df(\eta\circ\cW^{-1}\circ\eta)\geq \fr{2}{f}\br{df(\eta)}^2\quad \forall \cW,\eta\in T^{1,1}_g(M),}
such that $\cW$ is invertible and where $f=f(\cW).$

If $F$ is convex, then
\eq{df(\ad(\eta)\circ \cW^{-1}\circ\eta)\geq \fr 1pf^{-1}\br{df(\eta)}^2\quad\forall \cW\in T^{1,1}_g(M)~\forall \eta\in T^{1,1}(M)}
and
\eq{d^2f(\eta,\eta)\geq\fr{p-1}{p}f^{-1}\br{df(\eta)}^2\quad\forall \cW,\eta\in T^{1,1}_g(M),}
where we again assume that $\cW$ is invertible.
}

\pf{
Let $F$ be inverse concave. We first prove the case $p=1$. We have for $\cW,T\in T^{1,1}_g(M)$ with $\cW$ invertible, that
\eq{d\~f(T)=\~f^2 df(\cW^{-1}\circ T\circ \cW^{-1})}
and
\eq{d^2\~f(T,T)&=2\~f^3\br{df(\cW^{-1}\circ T\circ\cW^{-1})}^2\\
				&\hp{=}-\~f^2d^2f(\cW^{-1}\circ T\circ\cW^{-1})(\cW^{-1}\circ T\circ\cW^{-1})\\
                &\hp{=}-2\~f^2df(\cW^{-1}\circ T\circ\cW^{-1}\circ T\circ \cW^{-1})}
where $\~f$ is evaluated at $\cW$ and $f$ is evaluated at $\cW^{-1}$. By the assumption $d^2\~f\leq 0$ we obtain, also setting $T=\cW\circ\eta\circ \cW$ for $\eta\in T^{1,1}_g(M)$,
\eq{d^2f(\eta,\eta)+2df(\eta\circ \cW\circ\eta )\geq \fr{2}{f}\br{df(\eta)}^2.}
Also recalling that $f=f(\cW^{-1}),$ we obtain the result in case $p=1.$ For $p\neq 0$ simply calculate
\eq{\label{CF-1}df(\eta)=|p|F^{p-1}dF(\eta),}
\eq{d^2f(\eta,\eta)&=|p|(p-1)F^{p-2}(dF(\eta))^2+|p|F^{p-1}d^2F(\eta,\eta)\\
				&\geq\fr{p+1}{p}f^{-1}(df(\eta))^2-2|p|F^{p-1}dF(\eta\circ\cW^{-1}\circ\eta)\\
                &=\fr{p+1}{p}f^{-1}(df(\eta))^2-2df(\eta\circ\cW^{-1}\circ\eta).}
Now let $F$ be convex. The second inequality simply follows from the previous calculation. To prove the first inequality, we apply an idea from \cite[Thm.~2.3]{Andrews:/2007}, also compare the similar proof in \cite[Lemma~14]{BryanIvakiScheuer:12/2015}. Suppose first that $p=1$ and note that for each invertible $\cW\in T^{1,1}_g(M)$ the kernel $\mc{K}$ of the linear map
\eq{df_{\cW}\cn T^{1,1}(M)\ra \R}
has dimension $n^2-1$, due to
\eq{df(\cW)=\cW.}
Now let $\eta\in T^{1,1}(M)$, then there exists a decomposition
\[\eta=a\cW+\xi,\]
where we may assume wlog $a=1.$ Hence
\eq{df(\ad(\eta)\circ \cW^{-1}\circ\eta)&=df(\eta)+df(\ad(\xi))+df(\ad(\xi)\circ\cW^{-1}\circ\xi)\\
			&\geq df(\eta),}
since $df$ and $\cW$ can be diagonalized simultaneously. The result in case $p=1$ follows from $f=df(\cW)=df(\eta).$
The case $p\neq 0$ follows from the case $p=1$ and \eqref{CF-1}.
}

The normal variation speeds we consider do not only depend on the curvature, but are of a more general form. We collect our assumptions as follows.

\Theo{ass}{SpeedAss}{
We assume that $f$ is a nowhere vanishing velocity which is of the form
\eq{\label{Speed}f\cn \R_{+}\x UN &\x T^{1,1}(M)\ra \R\\ f(s,\nu,\cW)&=\p(s)\psi(\nu)\mrm{sgn}(p)F^{p}(\cW),}
where
\begin{itemize}
\item $p\neq 0$,
\item$\p\in C^{\8}(\R_{+})$ is a positive function acting on the support function $s$ and such that $\p\equiv 1$ if $N$ is neither the Euclidean nor the Minkowski space,
\item$\psi\in C^{\8}(UN)$ is a positive function defined on the unit bundle of $N$, such that $\psi$ is invariant under parallel transport and
\item $F$ is a strictly monotone, $1$-homogeneous curvature function which is
\begin{enumerate}
\item inverse concave, if $N=\R^{n+1}$ or $N=\R^{n,1}$ and
\item convex, otherwise.
\end{enumerate}
\end{itemize}
Here $\Sigma^{0,2}_{+}(M)\sub T^{0,2}M$ is the subbundle of positive definite tensors of type $(0,2)$, $UN$ is the unit sphere bundle on $N$ (including timelike unit vectors).
}


We define the parabolic operator $\mathcal{L}$ on smooth functions as follows
\[\mathcal{L}v=\dot{v}-f^{ij}v_{;ij}-T\ast\nabla v.\]
Here $T\ast \nabla v$ denotes any linear combination of arbitrary contractions of tensors $T$ with $\n v.$
\section{Evolution equations}
\subsection{Invariant calculations with respect with to affine connection}
We begin by collecting some basic evolution equations. The final aim is to deduce the evolution equation for the function
\eq{u=\fr{\dot{f}}{f}.}
 Recall that we are considering the flow equation
\eq{\label{Flow2}\dot{x}=-\s f\nu-x_{*}V,}
where
\eq{\label{EvEq-1}V=\grad_{h}f}
is the spatial gradient of $f$ with respect to the second fundamental form:
\eq{h(V,X)=Xf\quad\forall X\in \mc{T}M.}
Note that
\eq{\label{EvEq-2}\-\n_{X}\dot{x}&=-\s\-\n_{X}\br{f\nu}-\-\n_{X}V\\
				&=-\s h(V,X) \nu-\s f\-\n_{X}\nu-x_{*}\n_{X}V+\s h(X,V)\nu}
is tangential and hence we may define an endomorphism $A\in \mc{T}^{1,1}M$ by
\eq{\label{A} x_{*}(A(X))=-\-\n_{X}\dot{x}.}
Since we are dealing with strictly convex hypersurfaces, the symmetric bilinear form
\eq{\~g=\fr{h}{f}}
defines a metric tensor and we obtain a bilinear form associated with $A$:
\eq{B(X,Y)=\~g(X,A(Y)).} Note that
\eq{V=\grad_{\~g}\log |f|.}
Also let
\eq{\L(X,Y)=\overline{\mrm{Rm}}(\dot{x},X,\nu,Y).}
Note that $B$ is generally \emph{not} symmetric.
\Theo{lemma}{SymmetriesB}{
There holds
\eq{B(X,Y)&=B(Y,X)+\fr{1}{f}\overline{\mrm{Rm}}(\dot{x},\nu,X,Y)\\
		&=B(Y,X)+\fr 1f \L(X,Y)-\fr 1f\L(Y,X).}
}
\pf{
For $X,Y\in \mc{T}M,$
\eq{fB(X,Y)=h(X,A(Y))&=\s f\-g(\-\n_{Y}\nu,\-\n_{X}\nu)+\-g(\-\n_{X}\nu,\n_{Y}V)\\
					&=\s f\-g(\-\n_{X}\nu,\-\n_{Y}\nu)+h(X,\n_{Y}V).}
Moreover, we have
\eq{h(X,\n_{Y}V)&=YXf-h(\n_{Y}X,V)-\n h(V,X,Y)\\
			&=XYf-h(\n_{X}Y,V)-\n h(V,Y,X)+\overline{\mrm{Rm}}(\nu,V,X,Y)\\
			&=h(Y,\n_{X}V)+\overline{\mrm{Rm}}(\nu,V,X,Y).}
Hence the claim follows from the first Bianchi identity.
}
\subsubsection{Basic evolution equations}
\Theo{lemma}{BasicEv}{
Along the flow \eqref{Flow} there hold
\eq{\label{BE-g}\dot{g}=-2A^{\flat}_{\mrm{sym}},}
\eq{\label{BE-n}\fr{\-\n}{dt}\nu =0,}
\eq{\label{BE-W}\dot{\mc{W}}=A\circ \mc{W}+\L^{\sharp},}
\eq{\label{BE-h}\dot{h}=-fB+\L,}
\eq{\label{BE-tildeg}\dot{\~g}=-B-\~gu+\fr{\L}{f},}
\eq{\label{BE-gradf}\dot{V}&=\widetilde{\grad}u+A(V)+uV-\fr{1}{f}(\L^{t})^{\~\sharp}V,}
\eq{\label{BE-speed} \fr{\-\n}{dt}\dot{x}=u\dot{x}-x_{\ast}(\widetilde{\grad}u)+\fr 1f x_{\ast}((\L^{t})^{\~\sharp}V ).}
}
\pf{Let $X,Y\in\mc{T}M.$\newline
``\eqref{BE-g}'': By \eqref{EvEq-2} we have
\eq{\del_{t}\br{x^{*}\-g}(X,Y)&=\del_{t}\br{\-g(x_{*}X,x_{*}Y)}\\
					&=\-g(\-\n_{X}\dot{x},x_{*}Y)+\-g(x_{*}X,\-\n_{Y}\dot{x})\\
					&=-\-g(x_{*}A(X),x_{*}Y)-\-g(x_{*}A(Y),x_{*}X)\\
					&=-g(A(X),Y)-g(X,A(Y)).}
``\eqref{BE-n}'': We have $0=\del_{t}\bar{g}(\nu,\nu)$. Since $0=\del_{t}\bar{g}(\nu,x_{*}X)$, we get
\eq{\bar{g}(\-\n_{\dot{x}} \nu,X)=-\bar{g}(\nu,\-\n_{\dot{x}}X)=-\bar{g}(\nu,\-\n_{X}\dot{x})=0.}
``\eqref{BE-W}'': Recall that
\eq{\-\n_{x_{*}X}\nu=x_{*}\mc{W}(X).} Therefore, using (\ref{BE-n}), we calculate
\eq{\overline{\mrm{Rm}}(\dot{x},X)\nu=\-\n_{x_{*}\mc{W}(X)}\dot{x}+[\dot{x},x_{*}\mc{W}(X)]=-x_{*}(A(\mc{W}(X)))+x_{*}\dot{\mc{W}}(X).}
``\eqref{BE-h}'': Differentiate the Weingarten \eqref{Weingarten} equation and use \cref{SymmetriesB} to obtain
\eq{\del_{t}h(X,Y)=&\-g\br{\-\n_{\dot{x}}\-\n_{X}\nu,Y}+\-g\br{\-\n_{X}\nu,\-\n_{\dot{x}}Y}\\			=&\overline{\mrm{Rm}}\br{\dot{x},X,\nu,Y}-h(X,A(Y))\\
=&\L(X,Y)-fB(X,Y).}
``\eqref{BE-tildeg}'': It follows directly from \eqref{BE-h}.\newline
``\eqref{BE-gradf}'':
\eq{Xu&=X\del_{t}\log|f|\\	&=\del_{t}\br{\~g(X,\widetilde{\grad}\log|f|)}=\dot{\~g}(X,V)+\~g(X,\dot{V})\\
	&=-B(X,V)+\fr{1}{f}\L(X,V)-\~g(X,V)u+\~g(X,\dot{V})\\
	&=-\~g(X,A(V))+\fr{1}{f}\L(X,V)-\~g(X,V)u+\~g(X,\dot{V}).}
``\eqref{BE-speed}'':
\eq{\fr{\-\n}{dt}\dot{x}=&-\s\dot{f}\nu-\fr{\-\n}{dt}(x_{\ast}V)\\
				=&-\s\dot{f}\nu-\-\n_{x_{\ast}V}\dot{x}-[\dot{x},x_{\ast}V]\\
				=&-\s\dot{f}\nu+x_{\ast}(A(V))-x_{\ast}\dot{V}\\
				=&-\s uf\nu+x_{\ast}(A(V))-x_{\ast}(\widetilde{\grad}u)-x_{\ast}(A(V))-ux_{\ast}V+x_{\ast}(\fr{1}{f}(\L^{t})^{\~\sharp}V)\\
				=&u\dot{x}-x_{\ast}(\widetilde{\grad}u)+\fr 1f x_{\ast}((\L^{t})^{\~\sharp}V)
.}}
\subsubsection{Evolution equations involving the affine connection}
From now on, to simplify the calculations, we will work with the {\it{affine connection}} $\~\n$ induced by the transversal vector field $\dot{x}.$ Namely, for $X,Y\in\mc{T}M$ we have the decomposition
\eq{\-\n_{X}Y=x_{\ast}(\hat\n_{X}Y)+\~g(X,Y)\dot{x}.}
Note that $\hat{\n}$ is not the Levi-Civita connection for the so-called {\it{affine metric}} $\~g.$ Let $\~{\n}$ denote the Levi-Civita connection of $\~g$ and we define the difference tensor $D$ of type $(1,2)$ by
\eq{D_{X}Y=\hat{\n}_{X}Y-\~{\n}_{X}Y.}
Since both $\tilde{\nabla}$ and $\hat{\nabla}$ are torsion free, we have $D_XY=D_YX.$
The deviation of $\hat{\n}$ from the Levi-Civita connection of $\~g$ is measured by the {\it{cubic tensor}}
\eq{C(X,Y,Z):=& -\fr 12(\hat{\n}_{X}\~g)(Y,Z)\\
			=&\~g(D_{X}Y,Z)-\fr 12 \left[\overline{\mrm{{Rm}}}(X, Y)Z + \overline{\mrm{Rm}}(X, Z)Y\right]^{\dot{x}},
		}
where the superscript $\dot{x}$ denotes the $\dot{x}$ component in the splitting, $x^{\ast} \mathcal{T}N \simeq \mathcal{T}M \oplus \R \dot{x}$;
see \cite[Prop.~4.1]{NomizuSasaki:/1994} adjusted to the Riemannian setting.
\Theo{lemma}{AE-A}{
\eq{\label{AE-A} \dot{A}=A^{2}+uA+\~{\n}\widetilde{\grad}u+D \widetilde{\grad}u-\n \br{\fr 1f (\L^{t})^{\~\sharp}V}+\br{\overline{\mrm{Rm}}(\cdot,\dot{x})\dot{x}}^{\top}. }
%\eq{\label{AE-B} \dot{B}&=\~{\n}^{2}_{\~g}u+ C(\cdot,\widetilde{\grad}u,\cdot)+\fr 1f \L(A(\cdot),\cdot)\\
%			&\hp{=}-\~g\br{\n\br{\fr 1f \L^{\~\sharp}V},\cdot}+\~g(\br{\overline{\mrm{Rm}}(\cdot,\dot{x})\dot{x}}^{\top},\cdot)}
}
\pf{Let $X,Y\in \mc{T}M.$ Differentiate \eqref{A} with respect to $\dot{x}$ to obtain
\eq{\-\n_{\dot{x}}x_{\ast}(A(X))=-\-\n_{\dot{x}}\-\n_{X}\dot{x},}
\eq{[\dot{x},x_{\ast}(A(X))]+\-\n_{x_{\ast}(A(X))}\dot{x}=-\-\n_{X}\-\n_{\dot{x}}\dot{x}+\overline{\mrm{Rm}}(X,\dot{x})\dot{x}.}
Thus using (\ref{BE-speed}) we get
\eq{&x_{\ast}(\dot{A}(X))-x_{\ast}(A^{2}(X))-\overline{\mrm{Rm}}(X,\dot{x})\dot{x}\\
=&-\-\n_{X}\br{u\dot{x}-x_{\ast}(\widetilde{\grad}u)+\fr 1f x_{\ast}((\L^{t})^{\~\sharp}V)}\\
			=&-\br{\-\n_{X}u}\dot{x}+ux_{\ast}(A(X))+x_{\ast}(\hat{\n}_{X}\widetilde{\grad}u)+\~g(X,\widetilde{\grad}u)\dot{x}-\-\n_{X}\br{\fr 1f x_{\ast}((\L^{t})^{\~\sharp}V)}\\
			=&ux_{\ast}(A(X))+x_{\ast}(\~{\n}_{X}\widetilde{\grad}u)+x_{\ast}(D_{X}\widetilde{\grad}u)-\-\n_{X}\br{ x_{\ast}(\fr 1f(\L^{t})^{\~\sharp}V)}\\
			=&ux_{\ast}(A(X))+x_{\ast}(\~{\n}_{X}\widetilde{\grad}u)+x_{\ast}(D_{X}\widetilde{\grad}u)-x_{\ast}(\n_{X}\br{\fr 1f(\L^{t})^{\~\sharp}V})\\
			&+\s h(X,\fr 1f(\L^{t})^{\~\sharp}V)\nu.}
		}	
%``\eqref{AE-B}'': From \eqref{BE-tildeg} and \eqref{AE-A} we calculate
%\eq{\dot{B}(X,Y)&=\fr{d}{dt}\~g(A(X),Y)\\
%			&=-B(A(X),Y)-u\~g(A(X),Y)+\fr 1f \L(A(X),Y)\\
%			&\hp{=}+\~g(A^{2}(X),Y)+u\~g(A(X),Y)+\~g(\~{\n}_{X}\widetilde{\grad}u,Y)\\
%			&\hp{=}+\~g(D_{X}\widetilde{\grad}u,Y)-\~g\br{\n_{X}\br{\fr 1f \L^{\~\sharp}V},Y}\\
%			&\hp{=}+\~g(\br{\overline{\mrm{Rm}}(X,\dot{x})\dot{x}}^{\top},Y)\\
%			&=\~{\n}^{2}_{\~g}u(X,Y)+ C(X,\widetilde{\grad}u,Y)+\fr 1f \L(A(X),Y)\\
%			&\hp{=}-\~g\br{\n_{X}\br{\fr 1f \L^{\~\sharp}V},Y}+\~g(\br{\overline{\mrm{Rm}}(X,\dot{x})\dot{x}}^{\top},Y)}
\Theo{lemma}{Ev-Lambda}{
\eq{g(\partial_t\br{\fr{\L^{\sharp}}{f}}(X),Y)=&\fr 1f g(A(\L^{\sharp}(X)),Y)-\fr 1f \L(A(X),Y)-\fr 1f \overline{\mrm{Rm}}(\widetilde{\grad}u, X,\nu, Y)\\
&+\fr{1}{f^{2}}\overline{\mrm{Rm}}(\br{\L^{t}}^{\~\sharp}V,X,\nu,Y)+\fr{1}{f}\-\n_{\dot{x}}\overline{\mrm{Rm}}(\dot{x},X,\nu,Y).}
}
\pf{Differentiating the defining equation
\eq{g(\fr{\L^{\sharp}}{f}(X),Y)=\fr 1f\L(X,Y)=\fr 1f\overline{\mrm{Rm}}(\dot{x},X,\nu,Y)}
 with respect to $\dot{x}$ and using
\eq{2A^{\flat}_{\mrm{sym}}(\L^{\sharp}(X),Y)&=g(A(\L^{\sharp}(X)),Y)+g(A(Y),\L^{\sharp}(X))}
as well as \eqref{A} and \eqref{BE-speed} yield the result.
}
\Theo{lemma}{Lambdasharptilde}{
\eq{g(\n_{Z}\br{\fr{(\L^{t})^{\~\sharp}V}{f}},Y)=&g(\fr{(\L^{t})^{\~\sharp}V}{f},\n_{Z}Y)-\-\n_{Z}\overline{\mrm{Rm}}(\dot{x},\cW^{-1}(Y),\nu,\dot{x})\\
	&+\overline{\mrm{Rm}}(\dot{x},\cW^{-1}(Y),\dot{x},\cW(Z))-\overline{\mrm{Rm}}(\dot{x},\n_{Z}(\cW^{-1}(Y)),\nu,\dot{x})\\
    &+\s h(Z,\cW^{-1}(Y))\overline{\mrm{Rm}}(\dot{x},\nu,\nu,\dot{x})
	\\
	&+2\L(\cW^{-1}(Y),A(Z))-\L(A(Z),\cW^{-1}(Y)).}
}
\pf{Covariant differentiating the equation
\eq{\label{Lambdasharptilde-2}g(\fr{(\L^{t})^{\~\sharp}V}{f},Y)=\fr{h}{f}((\L^{t})^{\~\sharp}V,\cW^{-1}(Y))
=\L(\cW^{-1}(Y),V)=-\overline{\mrm{Rm}}(\dot{x},\cW^{-1}(Y),\nu,\dot{x})}
with respect to $Z$ gives
\eq{\label{Lambdasharptilde-1}g(\n_{Z}\br{\fr{(\L^{t})^{\~\sharp}V}{f}},Y)
	=&-g(\fr{(\L^{t})^{\~\sharp}V}{f},\n_{Z}Y)-\-\n_{Z}\overline{\mrm{Rm}}(\dot{x},\cW^{-1}(Y),\nu,\dot{x})\\
	&+\overline{\mrm{Rm}}(A(Z),\cW^{-1}(Y),\nu,\dot{x})-\overline{\mrm{Rm}}(\dot{x},\-\n_{Z}(\cW^{-1}(Y)),\nu,\dot{x})\\
			&-\overline{\mrm{Rm}}(\dot{x},\cW^{-1}(Y),\cW(Z),\dot{x})+\overline{\mrm{Rm}}(\dot{x},\cW^{-1}(Y),\nu,A(Z)).}
Moreover, by the Weingarten equation
\eq{\-\n_{Z}(\cW^{-1}(Y))=\n_{Z}(\cW^{-1}(Y))-\s h(Z,\cW^{-1}(Y))\nu.}
Inserting this last relation as well as \eqref{Flow2} into \eqref{Lambdasharptilde-1} gives the result.
}
We need one more lemma before calculating the main the evolution equation.
\Theo{lemma}{MixedTrace}{
There holds
% \eq{\mc{W}\circ A^2=\ad(A)\circ\L^{\sharp}+\L^{\sharp}\circ A+\ad(A)\circ\cW\circ A.}
\eq{\fr{1}{f}\operatorname{Tr}(d_{\cW}f\circ\cW\circ A^{2})&=\fr 1f\operatorname{Tr}_{d_{h}f}\L(\cdot,A(\cdot))-\fr 1f\operatorname{Tr}_{d_{h}f}\L(A(\cdot),\cdot)\\
&\hp{=}+\fr 1f\operatorname{Tr}_{d_{h}f}h(A(\cdot),A(\cdot)).}
}
\pf{The claim follows from Lemma \ref{SymmetriesB}:
\eq{\operatorname{Tr}(d_{\cW}f\circ\cW\circ A^{2})&=\operatorname{Tr}_{d_{h}f}h(\cdot,A^{2}(\cdot))=f\operatorname{Tr}_{d_{h}f}B(\cdot,A(\cdot))\\		&=f\operatorname{Tr}_{d_{h}f}B(A(\cdot),\cdot)+\operatorname{Tr}_{d_{h}f}\L(\cdot,A(\cdot))-\operatorname{Tr}_{d_{h}f}\L(A(\cdot),\cdot)\\
					&=\operatorname{Tr}_{d_{h}f}h(A(\cdot),A(\cdot))+\operatorname{Tr}_{d_{h}f}\L(\cdot,A(\cdot))-\operatorname{Tr}_{d_{h}f}\L(A(\cdot),\cdot).}
}
Now we are prepared to deduce the evolution equation of $u.$ Note that the next evolution equation will be the \textit{first} place where the actual domain in the definition of $f$ will play a role. In the following lemma, we assume
\eq{f(s,\nu,h_{ij},g_{ij})=\p(s)F(\nu,h_{ij},g_{ij}).}
\Theo{lemma}{Ev-u-new}{Under the flow \eqref{Flow} we have
\eq{\label{Ev-u}\dot{u}&-\operatorname{Tr}_{d_{h}f}\widetilde{\operatorname{Hess}}u-\operatorname{Tr}_{d_{h}f}C(\cdot,\widetilde{\grad}u,\cdot)+\fr 1f \operatorname{Tr}_{d_{h}f}\overline{\mrm{Rm}}(\widetilde{\grad}u,\cdot,\nu,\cdot)\\
&+\fr 12 \operatorname{Tr}_{d_hf} \left[\overline{\mrm{{Rm}}}(\cdot,\widetilde{\grad}u)(\cdot)+ \overline{\mrm{Rm}}(\cdot, \cdot)\widetilde{\grad}u\right]^{\dot{x}}\\
=&(\log\p)'' \dot{s}^{2}+(\log\p)'\ddot{s}+(\log\p)'d_{\cW}f(\dot{\cW})+\fr 1f d^{2}_{\cW}f(\dot{\cW},\dot{\cW})\\
	\hp{=}&+\fr 2f\operatorname{Tr}_{d_{h}f}h(A(\cdot),A(\cdot))+\fr 2f \operatorname{Tr}(d_{\cW}f\circ (A\circ\L^{\sharp}-\L^{\sharp}\circ A))\\
	\hp{=}&+\s\br{1-\fr{\operatorname{Tr}_{d_{h}f}h(\cdot,\cdot)}{f}}\overline{\mrm{Rm}}(\dot{x},\nu,\nu,\dot{x})+\fr 2f \operatorname{Tr}_{d_{h}f}\overline{\mrm{Rm}}(\cdot,\dot{x},\dot{x},\cW(\cdot))\\
&+\fr 1f\operatorname{Tr}_{d_{h}f}\-\n\overline{\mrm{Rm}}(\dot{x},\cdot,\nu,\cdot,\dot{x})+\fr 1f\operatorname{Tr}_{d_{h}f}\-\n\overline{\mrm{Rm}}(\dot{x},\cdot,\nu,\dot{x},\cdot).}
}
\pf{Note that
\eq{u=\fr{\dot{f}}{f}=(\log \p)'\dot{s}+\fr 1fd_{\cW}f(\dot{\cW})=(\log\p)'\dot{s}+\fr 1f d_{\cW}f(A\circ\cW+\L^{\sharp}).}
Hence taking time derivative of both sides we arrive at
\eq{\dot{u}=&(\log\p)''\dot{s}^{2}+(\log\p)'\ddot{s}+(\log\p)'\dot{s}d_{\cW}f(\dot{\cW})\\
		&-\fr{u}{f}d_{\cW}f(A\circ\cW+\L^{\sharp})+\fr 1f d^{2}_{\cW}f(\dot{\cW},\dot{\cW})+\fr 1f d_{\cW}f(\dot{A}\circ\cW+A\circ\dot{\cW}+\del_{t}\L^{\sharp})\\
		=&(\log\p)''\dot{s}^{2}+(\log\p)'\ddot{s}+(\log\p)'\dot{s}d_{\cW}f(\dot{\cW})+\fr 1f d^{2}_{\cW}f(\dot{\cW},\dot{\cW})\\
		&+\fr 1f d_{\cW}f(\dot{A}\circ\cW+A^{2}\circ\cW+A\circ\L^{\sharp}-uA\circ\cW)+d_{\cW}f\br{\del_{t}(\fr{{\L^{\sharp}}}{f} )}.}
Since
$d_{\cW}f(T)=\operatorname{Tr}(d_{\cW}f\circ T)$
and $d_{\cW}f$ commutes with $\cW$, we obtain
\eq{\dot{u}=&(\log\p)''\dot{s}^{2}+(\log\p)'\ddot{s}+(\log\p)'\dot{s} d_{\cW}f(\dot{\cW})+\fr 1f  d^{2}_{\cW}f(\dot{\cW},\dot{\cW})\\
	&+\fr 1f\operatorname{Tr}(d_{\cW}f\circ\cW\circ(\dot{A}+A^{2}-uA))+d_{\cW}f(\fr{A\circ\L^{\sharp}}{f}+\del_{t}\br{\fr{{\L^{\sharp}}}{f} })\\
		=&(\log\p)''\dot{s}^{2}+(\log\p)'\ddot{s}+(\log\p)'\dot{s} d_{\cW}f(\dot{\cW})+\fr 1f  d^{2}_{\cW}f(\dot{\cW},\dot{\cW})\\
	&+\fr{2}{f}\operatorname{Tr}(d_{\cW}f\circ\cW\circ A^{2})+\fr 1f\operatorname{Tr}\br{d_{\cW}f\circ\cW\circ\br{\~{\n}\widetilde{\grad}u+D \widetilde{\grad}u}}\\
	&-\fr 1f \operatorname{Tr}\br{d_{\cW}f\circ\cW\circ\n\br{\fr 1f (\L^{t})^{\~\sharp}V}}+d_{\cW}f(\fr{A\circ\L^{\sharp}}{f}+\del_{t}\br{\fr{{\L^{\sharp}}}{f} })\\
	&+\fr 1f \operatorname{Tr}\br{d_{\cW}f\circ\cW\circ\br{\overline{\mrm{Rm}}(\cdot,\dot{x})\dot{x}}^{\top}}.}
Rewriting the $\widetilde{\grad}$-terms we obtain
\eq{\label{Ev-u-5-new}\dot{u}&-\operatorname{Tr}_{d_{h}f}\~{\n}^{2}_{\~g}u-\operatorname{Tr}_{d_{h}f}C(\cdot,\widetilde{\grad}u,\cdot)\\
		&+\fr 12 \operatorname{Tr}_{d_hf} \left[\overline{\mrm{{Rm}}}(\cdot, \widetilde{\grad}u)(\cdot)+ \overline{\mrm{Rm}}(\cdot, \cdot)\widetilde{\grad}u\right]^{\dot{x}}\\
	=&(\log\p)''\dot{s}^{2}+(\log\p)'\ddot{s}+(\log\p)'\dot{s} d_{\cW}f(\dot{\cW})+\fr 1f  d^{2}_{\cW}f(\dot{\cW},\dot{\cW})\\
	&+\fr 1f \operatorname{Tr}\br{d_{\cW}f\circ\cW\circ\br{\overline{\mrm{Rm}}(\cdot,\dot{x})\dot{x}}^{\top}}+\fr{2}{f}\operatorname{Tr}(d_{\cW}f\circ\cW\circ A^{2})\\
	&-\fr 1f \operatorname{Tr}\br{d_{\cW}f\circ\cW\circ\n\br{\fr 1f (\L^{t})^{\~\sharp}V}}+d_{\cW}f(\fr{A\circ\L^{\sharp}}{f}+\del_{t}\br{\fr{{\L^{\sharp}}}{f} }).}		
Using the formulas from \cref{Ev-Lambda}, \cref{Lambdasharptilde} and \cref{MixedTrace}, we treat the last three terms of (\ref{Ev-u-5-new}) in order.
\begin{itemize}
  \item From \cref{MixedTrace} we obtain
\eq{\label{Ev-u-4-new}\fr{2}{f}\operatorname{Tr}(d_{\cW}f\circ\cW\circ A^{2})=&\fr 2f\operatorname{Tr}_{d_{h}f}\L(\cdot,A(\cdot))-\fr 2f\operatorname{Tr} (d_Wf\circ\L^{\sharp}\circ A)\\						&+\fr 2f\operatorname{Tr}_{d_{h}f}h(A(\cdot),A(\cdot)),}
where we used
\begin{align}
\operatorname{Tr}_{d_{h}f}\L(A(\cdot),\cdot)=\operatorname{Tr}_{d_hf}g(\L^{\sharp}\circ A(\cdot),\cdot)=\operatorname{Tr} (d_{\mc{W}}f\circ\L^{\sharp}\circ A).
\end{align}
  \item \cref{Lambdasharptilde} implies that
\eq{\label{Ev-u-2-new}&-\fr 1f \operatorname{Tr}\br{d_{\cW}f\circ\cW\circ\n\br{\fr 1f (\L^{t})^{\~\sharp}V}}\\
	=~&\fr{1}{f}\operatorname{Tr}_{d_{h}f}\L(A(\cdot),\cdot)-\fr{2}{f}\operatorname{Tr}_{d_{h}f}\L(\cdot,A(\cdot))-\fr{\s}{f}\operatorname{Tr}_{d_{h}f}h(\cdot,\cdot)\overline{\mrm{Rm}}(\dot{x},\nu,\nu,\dot{x})\\
	\hp{=}~&-\fr{1}{f}\operatorname{Tr}_{d_{h}f}\overline{\mrm{Rm}}(\dot{x},\cdot,\dot{x},\cW(\cdot))+\fr 1f \operatorname{Tr}_{d_{h}f}\-\n\overline{\mrm{Rm}}(\dot{x},\cdot,\nu,\dot{x},\cdot)\\
	\hp{=}~&+\fr 1f \operatorname{Tr}_{d_{h}f\ast\cW}\overline{\mrm{Rm}}(\dot{x},\n_{(\cdot)}(\cW^{-1}(\cdot)),\nu,\dot{x}).}
To treat the last term of $\eqref{Ev-u-2-new}$, we use the Codazzi equation:\footnote{$g(\n_X \mc{W}(\mc{W}^{-1}(Y)),Z)=\n h(\mc{W}^{-1}(Y),Z,X)=\n h(\mc{W}^{-1}(Y),X,Z)+\overline{\mrm{Rm}}(\nu,\mc{W}^{-1}(Y),X,Z);$ therefore,
$\n_X \mc{W}(\mc{W}^{-1}(Y))=(\n h(\mc{W}^{-1}(Y),X,\cdot))^{\sharp}+(\overline{\mrm{Rm}}(\nu,\cW^{-1}(Y))X)^{\top}.$}
\eq{\n_{X}(\cW^{-1}(Y))=&-\cW^{-1}(\n_{X}\cW(\cW^{-1}(Y)))+\cW^{-1}(\n_{X}Y)\\
=&-\cW^{-1}\br{(\n h(X,\cW^{-1}(Y),\cdot))^{\sharp}}+\cW^{-1}(\n_{X}Y)\\
&-\cW^{-1}(\overline{\mrm{Rm}}(\nu,\cW^{-1}(Y))X)^{\top}.}
Hence
\eq{\label{Ev-u-3-new}&\fr 1f \operatorname{Tr}_{d_{h}f\ast\cW}\overline{\mrm{Rm}}(\dot{x},\n_{(\cdot)}(\cW^{-1}(\cdot)),\nu,\dot{x})\\
			=~&-\fr{1}{f}\overline{\mrm{Rm}}(\dot{x},V,\nu,\dot{x})
				+\fr 1f \operatorname{Tr}_{d_{h}f}\L(\cW^{-1}(\overline{\mrm{Rm}}(\nu,\cdot)(\cdot)^{\top}),V)\\
			=~&\s\overline{\mrm{Rm}}(\dot{x},\nu,\nu,\dot{x})
				+\fr 1f \operatorname{Tr}_{d_{h}f}\L(\cW^{-1}(\overline{\mrm{Rm}}(\nu,\cdot)(\cdot)^{\top}),V).}
  \item In view of \cref{Ev-Lambda} we have
\eq{\label{Ev-u-1-new}&d_{\cW}f(\fr{A\circ\L^{\sharp}}{f}+\del_{t}\br{\fr{{\L^{\sharp}}}{f} })\\ =&\fr{2}{f}\operatorname{Tr} (d_{\cW}f\circ A\circ{\L^{\sharp}})-\fr{1}{f}\operatorname{Tr}(d_{\cW}f\circ\L^{\sharp}\circ A)-\fr 1f\operatorname{Tr}_{d_{h}f}\overline{\mrm{Rm}}(\widetilde{\grad}u,\cdot,\nu,\cdot)\\
\hp{=}&+\fr{1}{f^{2}}\operatorname{Tr}_{d_{h}f}\overline{\mrm{Rm}}((\L^{t})^{\~\sharp}V,\cdot,\nu,\cdot)+\fr 1f\operatorname{Tr}_{d_{h}f}\-\n\overline{\mrm{Rm}}(\dot{x},\cdot,\nu,\cdot,\dot{x}).}
Also, note that $(\L^{t})^{\~\sharp}=f\mc{W}^{-1}\circ(\L^{t})^{\sharp}$; therefore,
\eq{\fr{1}{f^{2}}\operatorname{Tr}_{d_{h}f}\overline{\mrm{Rm}}((\L^{t})^{\~\sharp}V,\cdot,\nu,\cdot)&=-\fr 1f \operatorname{Tr}_{d_{h}f}\overline{\mrm{Rm}}(\nu,\cdot,\cdot,\fr{(\L^{t})^{\~\sharp}V}{f})\\
		&=-\fr 1f \operatorname{Tr}_{d_{h}f}g(\fr{(\L^{t})^{\~\sharp}V}{f},\overline{\mrm{Rm}}(\nu,\cdot)(\cdot)^{\top})\\
        &=-\fr 1f \operatorname{Tr}_{d_{h}f}g(\mc{W}^{-1}\circ(\L^{t})^{\sharp}(V),\overline{\mrm{Rm}}(\nu,\cdot)(\cdot)^{\top})\\
		&=-\fr 1f \operatorname{Tr}_{d_{h}f}\L(\cW^{-1}(\overline{\mrm{Rm}}(\nu,\cdot)(\cdot)^{\top}),V).}
\end{itemize}
Putting these last three items all together gives
\eq{\label{Ev-u-4-new}&\fr{2}{f}\operatorname{Tr}(d_{\cW}f\circ\cW\circ A^{2})-\fr 1f \operatorname{Tr}\br{d_{\cW}f\circ\cW\circ\n\br{\fr 1f (\L^{t})^{\~\sharp}V}}\\
	&+d_{\cW}f(\fr{A\circ\L^{\sharp}}{f}+\del_{t}\br{\fr{{\L^{\sharp}}}{f} })\\
	=~& \fr 2f \operatorname{Tr}(d_{\cW}f\circ (A\circ \L^{\sharp}-\L^{\sharp}\circ A))-\fr 1f \operatorname{Tr}_{d_{h}f}\overline{\mrm{Rm}}(\widetilde{\grad}u,\cdot,\nu,\cdot)\\
	\hp{=}~&+\fr 1f\operatorname{Tr}_{d_{h}f}\-\n\overline{\mrm{Rm}}(\dot{x},\cdot,\nu,\cdot,\dot{x})+\s\br{1-\fr{\operatorname{Tr}_{d_{h}f}h(\cdot,\cdot)}{f}}\overline{\mrm{Rm}}(\dot{x},\nu,\nu,\dot{x})\\
	\hp{=}~&-\fr{1}{f}\operatorname{Tr}_{d_{h}f}\overline{\mrm{Rm}}(\dot{x},\cdot,\dot{x},\cW(\cdot))+\fr 1f \operatorname{Tr}_{d_{h}f}\-\n\overline{\mrm{Rm}}(\dot{x},\cdot,\nu,\dot{x},\cdot)+\fr 2f\operatorname{Tr}_{d_{h}f}h(A(\cdot),A(\cdot)).}
Inserting \eqref{Ev-u-4-new} into \eqref{Ev-u-5-new} gives the claimed result.
}
\section{Gauss Map and Duality}\label{Duality}
\label{gauss_duality}
In this section, we give a brief review of a duality relation between strictly convex hypersurfaces of the unit sphere $\S^{n+1}$ and a duality relation between strictly convex hypersurfaces of the hyperbolic space with such of the De Sitter space. The relevant results can be found in \cite[Ch.~9, 10]{Gerhardt:/2006}. For convenience, we will state the main results here.
\subsection{Duality in the sphere}
Let $x\cn M_0\ra M\hra \S^{n+1}$ be a strictly convex closed hypersurface. Let the {\it{Gauss map}} $\~x\in T_x(\R^{n+2})$ represent the unit normal vector to $M$, $\nu\in T_x(\S^{n+1}).$ Then the mapping
\eq{\label{GaussMap}\~x\cn M_0\ra \S^{n+1}}
is also the embedding of a closed and strictly convex hypersurface. The geometry of $\~x$ is governed by the following theorem:
\Theo{thm}{SphereDuality}{\cite[Thm.~9.2.5]{Gerhardt:/2006}
Let $x\cn M_0\ra M\ra \S^{n+1}$ be a closed, connected, strictly convex hypersurface of class $C^{m},$ $m\geq 3,$ then the Gauss map $\~x$ in \eqref{GaussMap} is the embedding of a closed, connected, strictly convex hypersurface $\~M\sub \mathbb{S}^{n+1}$ of class $C^{m-1}.$ Viewing $\~M$ as a codimension $2$ submanifold in $\R^{n+2},$ its Gaussian formula is
\eq{\~x_{;ij}=-\~g_{ij}\~x-\~h_{ij}x,}
where $\~g_{ij},$ $\~h_{ij}$ are the metric and the second fundamental form of the hypersurface $\~M\sub \mathbb{S}^{n+1}$ and $x=x(\xi)$ is the embedding of $M$ which also represents the exterior normal vector of $\~M$. The second fundamental form $\~h_{ij}$ is defined with respect to the interior normal vector.

The second fundamental forms of $M,$ $\~M$ and the corresponding principal curvatures $\k_{i},$ $\~\k_{i}$ satisfy
\eq{h_{ij}=\~h_{ij}=\ip{\~x_{;i}}{x_{;j}},\quad \~\k_{i}=\k_{i}^{-1}.}
}
We point out that $\tilde{M}$ is called the polar set to $M$ and it has the following elegant representation:
$$\tilde{M}=\{y\in \mathbb{S}^{n+1}: \sup_{y\in M}\langle x,y\rangle=0\},$$
where $\langle \cdot,\cdot\rangle$ is the inner product in $\mathbb{R}^{n+2};$ see \cite[Thm.~9.2.9]{Gerhardt:/2006}.
\subsection{Duality between hyperbolic space and De Sitter space} In this section, $\langle \cdot,\cdot\rangle$ denotes the inner product of $\mathbb{R}^{n+1,1}.$
The de Sitter space is the Lorentzian spaceform in the Minkowski space with constant sectional curvature $C=1:$
\eq{\mathbb{S}^{n,1}=\{x\in\mathbb{R}^{n+1,1}: \langle x,x\rangle=1 \},}
whereas the hyperbolic space is a Riemannian spaceform in the Minkowski space with constant sectional curvature $C=-1$:
\eq{\mathbb{H}^{n+1}=\{x\in\mathbb{R}^{n+1,1}: \langle x,x\rangle=-1,~ x^{0}>0\},}
where $x^0$ is the time coordinate.

Similarly, as for the sphere, given an embedding
\eq{x\cn M_0\ra M\sub\H^{n+1}}
of a closed and strictly convex hypersurface, the representation $\~x\in T_x(\R^{n+1,1})$ of the exterior normal vector $\nu\in T_x(\H^{n+1})$ yields the embedding
\eq{\label{HypGaussMap}\~x\cn M_0\ra \~M\sub \S^{n,1}}
of a strictly convex, closed and spacelike hypersurface $\~M.$
We also call this map $\~x$ the {\it{Gauss map of $M$}} and similarly as in the spherical case we have the following theorem:
\Theo{thm}{DeSitterDuality}{\cite[Thm.~10.4.4]{Gerhardt:/2006}
Let $x\cn M\ra \H^{n+1}$ be a closed, connected, strictly convex hypersurface of class $C^{m},$ $m\geq 3,$ then the Gauss map $\~x$ as in \eqref{HypGaussMap} is the embedding of a closed, spacelike, achronal, strictly convex hypersurface $\~M\sub \mathbb{S}^{n,1}$ of class $C^{m-1}.$ Viewing $\~M$ as a codimension $2$ submanifold in $\R^{n+1,1},$ its Gaussian formula is
\eq{\~x_{;ij}=-\~g_{ij}\~x+\~h_{ij}x,}
where $\~g_{ij},$ $\~h_{ij}$ are the metric and the second fundamental form of the hypersurface $\~M\sub \mathbb{S}^{n,1}$ and $x=x(\xi)$ is the embedding of $M$ which also represents the future directed normal vector of $\~M$. The second fundamental form $\~h_{ij}$ is defined with respect to the future directed normal vector, where the time orientation of $N$ is inherited from $\R^{n+1,1}$.

The second fundamental forms of $M,$ $\~M$ and the corresponding principal curvatures $\k_{i},$ $\~\k_{i}$ satisfy
\eq{h_{ij}=\~h_{ij}=\ip{\~x_{;i}}{x_{;j}},\quad \~\k_{i}=\k_{i}^{-1}.}
}
The hypersurface $\tilde{M}$ is called the polar set to $M$ and it can be represented as follows \cite[Thm.~10.4.8]{Gerhardt:/2006}:
$$\tilde{M}=\{y\in \mathbb{S}^{n,1}: \sup_{y\in M}\langle x,y\rangle=0\}.$$
In this model of the hyperbolic space the point $(1,0,\ldots,0)$ is called the {\it{Beltrami point}}. For a given strictly convex hypersurface $M\sub\H^{n+1}$, $M$ bounds a strictly convex body $\hat{M}$ of the hyperbolic space, cf.~\cite[Thm.~10.3.1]{Gerhardt:/2006}, and due to the homogeneity of the hyperbolic space, any point in $\hat{M}$ may act as Beltrami point after suitable ambient change of coordinates. Therefore, in addition to the statement of \cref{DeSitterDuality}, \cite[Thm.~10.4.9.]{Gerhardt:/2006} implies that the dual $\~M$ is contained in the future of the slice $\{x^0=0\}$,
\eq{\~M\sub \S^{n,1}_{+}=\{x\in \S^{n,1}\cn x^0>0\}.}
We will also need the reverse direction starting from a strictly convex, spacelike hypersurface in $\S^{n,1}.$
\Theo{thm}{HyperbolicDuality}{\cite[Thm.~10.4.5]{Gerhardt:/2006}
Let $x\cn \~M\ra \S^{n,1}$ be a closed, connected, spacelike, strictly convex hypersurface of class $C^{m},$ $m\geq 3,$ such that, when viewed as a codimension 2 submanifold in $\R^{n+1,1}$, its Gaussian formula is
\eq{\~x_{;ij}=-\~g_{ij}\~x+\~h_{ij}x,}
where $\~x=\~x(\xi)$ is the embedding, $x$ the future directed normal vector , and $\~g_{ij}$, $\~h_{ij}$ the induced metric and the second fundamental form of the hypersurface in $\S^{n,1}$. Then we define the Gauss map as $x=x(\xi)$
\eq{x\cn \~M\ra \H^{n+1}\sub\R^{n+1,1}.}
The Gauss map is the embedding of a closed, connected, strictly convex hypersurface $M$ in $\H^{n+1}.$
Let $\~g_{ij},$ $\~h_{ij}$ be the metric and the second fundamental form of $M$, then, when viewed as a codimension 2 submanifold, $M$ satisfies the relations
\eq{x_{ij}=g_{ij}x-h_{ij}\~x,}
\eq{h_{ij}=\~h_{ij}=\ip{x_{;i}}{\~x_{;j}},} and
\eq{\~\k_{i}=\k_{i}^{-1},}
where $\k_{i}$, $\~\k_{i}$ are the corresponding principal curvatures.
}
The following illustration shall give a clearer picture of the duality.
{\color{red}Include graphics}
\subsection{Dual flows}
For a curvature flow
\eq{\label{NormalFlow}\dot{x}=-\s f\nu,}
where $\s=\ip{\nu}{\nu}$, we want to derive the curvature flow equation of the Gauss maps $\~x$. In both cases the pair $x,\~x$ satisfies
\eq{\ip{x}{\~x}=0,}
where $\ip{\cdot}{\cdot}$ represents the Euclidean and the Minkowski inner product respectively for flows in $\S^{n+1}$ or in $\H^{n+1}$ and $\S^{n,1}$.
Hence
\eq{\ip{\dot{\~x}}{x}=-\ip{\~x}{\dot{x}}=-\ip{\~x}{- \s f\~x}= f.}
 Also, note that $\ip{\~x}{\~x_{;i}}=0$; therefore, due to the Weingarten equation \cite[Lem.~9.2.4, Lem.~10.4.3]{Gerhardt:/2006},
\eq{\ip{\dot{\~x}}{\~x_{;i}}=h^k_i\ip{\dot{\~x}}{x_{;k}}=-h^k_i\ip{\~x}{\dot{x}_{;k}}=h^k_if_{;k}.}
Since $x=\~\nu$ and $\~x_{;i}$ span $T_{\~x}(\S^{n+1})$ and $T_{\~x}(\S^{n,1})$ respectively, we obtain
\eq{\dot{\~x}&=\ip{x}{x} f\~\nu+h^k_mf_{;k}\~g^{ml}\~x_{;l}\\
			&=\~\s f\~\nu+\~b^k_m\~g^{ml}f_{;k}\~x_{;l}\\
            &=\~\s f\~\nu+\~b^{kl}f_{;k}\~x_{;l},}
where $f$ is still evaluated at $\mc{W}$. However, there holds
\eq{\label{DualFlow-3}f=f(\mc{W})=\fr{1}{f^{-1}(\~{\mc{W}}^{-1})}=\~f^{-1}(\~{\mc{W}})\equiv -\Phi(\~{\mc{W}})}
and hence
\eq{\label{DualHyperbolic}\dot{\~x}=-\~\s\Phi\~\nu-\~b^{kl}\Phi_{;k}\~x_{;l},}
where $\~\s=\ip{\~\nu}{\~\nu}$ and where $\Phi$ is now evaluated at the ``correct" Weingarten map $\~{\mc{W}}$. We see that the dual of a flow of the form
\eqref{NormalFlow} in $\S^{n+1}$ $\br{\H^{n+1},\S^{n,1}}$ moves precisely as a flow of the form \eqref{Flow} in $\S^{n+1}$ $\br{\S^{n,1},\H^{n+1}}$.
\section{Locally symmetric spaces}
In this section we prove the proposed Harnack inequalities. We restrict to locally symmetric spaces, since in more general settings we are not aware how to deal with the terms including derivatives of the Riemannian curvature tensor. In order to prove our main theorems, we need the following corollary of \eqref{Ev-u-new} with bonus term $\mc{B}$, which will be the basis for the proofs of all our main theorems.
\Theo{lemma}{Homogeneous}{
Let the ambient space $N$ be locally symmetric, i.e.,
\eq{\-\n\overline{\mrm{Rm}}=0.}
Let the speed $f$ satisfy \cref{SpeedAss}, let $\mc{B}\in\R$ and define
\eq{q=t(u-\mc{B})+\fr{p}{p+1}.}
Then for $p>0$, any strictly convex solution to  \eqref{Flow} satisfies
\eq{\label{Ev-q}\mc{L}q&\geq\fr{t}{\p^{2}}\br{\p''\p+\fr{(1-p)\p'^{2}}{p}}f^{2}+\fr{2t\s}{p}\fr{\p'}{\p}fu+\fr{p+1}{p}(u-\mc{B})q\\
			&\hp{=}-t\fr{p+1}{p}(u-\mc{B})^{2}+t\fr{1-p}{p}u^{2}+\fr{2t}{p}\br{u-\fr{f^{ij}\L_{ij}}{f}}^{2}\\
			&\hp{=}+t\s\br{1-\fr{f^{ij}h_{ij}}{f}}\-R_{\a\b\g\d}\dot{x}^{\a}\nu^{\b}\dot{x}^{\g}\nu^{\d}+\fr{2t}{f} f^{ij}h^{k}_{j}\-R_{\a\b\g\d}\dot{x}^{\a}x^{\b}_{;i}\dot{x}^{\g}x^{\d}_{;k},
}
For $p<0$ this inequality is reversed.
}
\pf{We have to distinguish two cases. First let $N=\R^{n+1}$ or $N=\R^{n,1}$.
Recall that the support function is given by $s=\s\ip{x}{\nu}.$
Hence
\eq{\dot{s}=\s\ip{\dot{x}}{\nu}=-\s f}
and
\eq{\ddot{s}&=-\s\dot{f}=\fr{\p'}{\p}f^2-\s ff^{ij}B_{ij}.}
We also have
\eq{\label{R=0-3} u=-\s \fr{\p'}{\p}f+f^{ij}B_{ij}.}
Combining these relations with \cref{CF}, in case $p>0$ \eqref{Ev-u} becomes
\eq{\label{R=0-1}\mathcal{L}u&=\br{\log\p}''\dot{s}^2+\br{\log\p}'\ddot{s}+\br{\log\p}'\dot{s}f^{ij}B_{ij}+f\br{2f^{il}b^{jk}+f^{ij,kl}}B_{ij}B_{kl}\\
				&\geq \br{\log\p}''\dot{s}^2+\br{\log\p}'\ddot{s}+\br{\log\p}'\dot{s}f^{ij}B_{ij}+\fr{p+1}{p}\br{f^{ij}B_{ij}}^{2}\\
				&=\fr{\p''}{\p}f^{2}-\fr{2\s\p'}{\p}ff^{ij}B_{ij}+\fr{p+1}{p}u^{2}+\fr{2(p+1)\s\p'}{p\p}fu+\fr{p+1}{p}\fr{\p'^{2}}{\p^{2}}f\\
				&=\fr{p+1}{p}u^{2}+\fr{1}{\p^{2}}\br{\p''\p+\fr{(1-p)\p'^{2}}{p}}f^{2}+\fr{2\s}{p}\fr{\p'}{\p}fu,}
with reversed inequality if $p<0$.
				
In case that $N$ is neither Euclidean nor Minkowski we use \cref{CF} and \eq{u=f^{ij}\left(B_{ij}+\frac{1}{f}\Lambda_{ij}\right)} to deduce
\eq{\label{convexity}&2ff^{il}b^{jk}B_{kl}B_{ij}+ff^{ij,kl}\br{B_{kl}+\fr 1f\L_{kl}}\br{B_{ij}+\fr 1f\L_{ij}}\\
&\geq \frac{2}{p} f^{ij}f^{kl}B_{kl}B_{ij}+\fr{p-1}{p}f^{ij}f^{kl}\br{B_{kl}+\fr 1f\L_{kl}}\br{B_{ij}+\fr 1f\L_{ij}}\\
&=\frac{p+1}{p}f^{ij}f^{kl}\br{B_{kl}+\fr 1f\L_{kl}}\br{B_{ij}+\fr 1f\L_{ij}}\\
&-\frac{2}{p}\frac{f^{kl}\L_{kl}}{f}\frac{f^{ij}\L_{ij}}{f}-\frac{4}{p} \frac{f^{kl}\L_{kl}}{f}f^{ij}\left(B_{ij}+\fr 1f\L_{ij}-\fr 1f\L_{ij}\right)\\
&=\frac{p+1}{p}u^2+\frac{2}{p}\left(\frac{f^{ij}\L_{ij}}{f}\right)^2-\frac{4}{p}\frac{f^{ij}\L_{ij}}{f}u,
}
with reversed inequality if $p<0$.
Therefore, from (\ref{convexity}) and \eqref{Ev-u-1} we deduce
\eq{\label{LocHom-1}\mathcal{L}u&\geq \frac{p+1}{p}u^2-\frac{4}{p}\frac{f^{ij}\L_{ij}}{f}u+\frac{2}{p}\left(\frac{f^{ij}\L_{ij}}{f}\right)^2
\\
&\hp{=}+\s\br{1-\fr{f^{ij}h_{ij}}{f}}\-R_{\a\b\g\d}\dot{x}^{\a}\nu^{\b}\dot{x}^{\g}\nu^{\d}+\fr 2f f^{ij}h^{k}_{j}\-R_{\a\b\g\d}\dot{x}^{\a}x^{\b}_{;i}\dot{x}^{\g}x^{\d}_{;k},
        }
if $p>0$ and the reversed inequality if $p<0$.

In all cases we obtain, using \eqref{R=0-1} and \eqref{LocHom-1},
\eq{\mc{L}q&=u-\mc{B}+t\mc{L} u\\
			&=\fr{p+1}{p}(u-\mc{B})q-t\fr{p+1}{p}(u-\mc{B})^{2}+t\mc{L}u\\
			&\geq\fr{t}{\p^{2}}\br{\p''\p+\fr{(1-p)\p'^{2}}{p}}f^{2}+\fr{2t\s}{p}\fr{\p'}{\p}fu+\fr{p+1}{p}(u-\mc{B})q\\
			&\hp{=}-t\fr{p+1}{p}(u-\mc{B})^{2}+t\fr{1-p}{p}u^{2}+\fr{2t}{p}\br{u-\fr{f^{ij}\L_{ij}}{f}}^{2}\\
			&\hp{=}+t\s\br{1-\fr{f^{ij}h_{ij}}{f}}\-R_{\a\b\g\d}\dot{x}^{\a}\nu^{\b}\dot{x}^{\g}\nu^{\d}+\fr{2t}{f} f^{ij}h^{k}_{j}\-R_{\a\b\g\d}\dot{x}^{\a}x^{\b}_{;i}\dot{x}^{\g}x^{\d}_{;k},
}
with reversed inequality of $p<0$.
}
With this lemma we prove our various variants of the Harnack inequality. We proceed from general to specific ambient spaces.
\subsection*{Euclidean and Minkowski space}
%\section{Euclidean and Minkowski space}
In the case that the ambient curvature vanishes, we obtain the following Harnack inequalities for anisotropic flows claimed in Theorem \ref{Euclidean}. In particular, the theorem includes and extend the well-known Harnack inequalities from \cite{Andrews:09/1994} in the Euclidean space and they are completely new in the Minkowski space.
\Theo{thm}{R=0}{
Let $N$ be either the Euclidean or the Minkowski space and let $F$, $\p$ and $\psi$ satisfy the assumptions of \cref{Euclidean}. Then along \eqref{Flow} there holds
\eq{tu+\fr{p+1}{p}\geq 0.}
}
\pf{Apply \eqref{Ev-q} with $\mc{B}=0$ to obtain that $q$ satisfies
\eq{\mc{L}q\geq \fr{2\s}{p}\fr{\p'}{\p}fq-\fr{2\s}{p+1}\fr{\p'}{\p}f+\fr{p+1}{p}uq}
with reversed inequality if $p<0$. If the ambient space is Euclidean, the maximum principle gives the Harnack estimate. If $N=\mathbb{R}^{n,1}$, due to our assumptions in Theorem \ref{Euclidean}, we can apply the maximum principle on the compact set $K$ and prove the claimed Harnack inequalities in each case.
}
\subsection*{Locally symmetric Einstein spaces of non-negative sectional curvature}
Here we obtain Harnack inequalities for the mean curvature flow:
\Theo{thm}{LocSymm}{
Suppose $N$ is a Riemannian locally symmetric Einstein space with non-negative sectional curvature. Let $f=H$. Then for any strictly convex solution to \eqref{Flow} there holds
\eq{\label{Bonus}t\br{u-\fr{\-R}{n+1}}+\fr{1}{2}\geq 0,}
where $\-R$ is the scalar curvature.
}
\pf{
(i)~We use \eqref{Ev-q} with
\eq{\mc{B}=\fr{\-R}{n+1},}
where $R$ is the scalar curvature. In this situation we have $\s=1$ and $f^{ij}h_{ij}=g^{ij}$ and hence the second line of \eqref{Ev-q} is non-negative. Furthermore there holds
\eq{f^{ij}\L_{ij}&=g^{ij}\-R_{\a\b\g\d}\dot{x}^{\a}x^{\b}_{;i}x^{\g}_{;j}\nu^{\d}\\
			&=-\-R_{\a\b}\dot{x}^{\a}\nu^{\d}\\
			&=\fr{\-R}{n+1}f.}
Hence the claim follows from the maximum principle applied to \eqref{Ev-q}.
}
\subsection*{The sphere}
In the case of the sphere, inequality \eqref{Bonus} is precisely the Harnack inequality with bonus term as it was already deduced in \cite{BryanIvaki:08/2015}.
We recover the class of speeds for which we could prove a Harnack inequality without bonus term in \cite{BryanIvakiScheuer:12/2015}.
\Theo{thm}{Sphere}{
Suppose $N=\S^{n+1}$ and $F$ is a monotone, convex and $1$-homogeneous curvature function. Let $0<p\leq 1$ and $f=F^p.$ Then for any strictly convex solution to \eqref{Flow} there holds
\eq{tu+\fr{p}{p+1}\geq 0.}
}
\pf{
Again the second line of \eqref{Ev-q} is non-negative. Also note that now we have
\eq{\L_{ij}=fg_{ij}}
We calculate
\eq{-t\fr{p+1}{p}u^{2}+t\fr{1-p}{p}u^{2}+2t\br{u-\fr{f^{ij}\L_{ij}}{f}}^{2}
&=-4tu\fr{f^{ij}\L_{ij}}{f}+2t\br{\fr{f^{ij}\L_{ij}}{f}}^{2}\\
&\geq-4q\fr{f^{ij}\L_{ij}}{f}+\fr{4p}{p+1}\fr{f^{ij}\L_{ij}}{f},
	}
and hence the maximum principle implies the claim again.

}
By applying the dual flow method developed in \Cref{Duality}, we obtain pseudo-Harnack inequalities
for a class of inverse curvature flows.
 \Theo{thm}{ExpandingSphere}{
Suppose $N=\S^{n+1}$ and $F$ is a monotone, inverse convex and $1$-homogeneous curvature function. Let $-1\leq p<0$ and $f=-F^{p}.$ Then for any strictly convex solution of \eqref{NormalFlow} we have
\eq{\del_t\br{ft^{\fr{p}{p-1}}}\leq 0.}
%In particular, this yields
%\eq{\del_tf+\fr{p}{p+1}\fr{f}{t}\leq 0.}
}
\pf{
The dual flow of \eqref{DualHyperbolic} with speed
\eq{\Phi(\~{\mc{W}})=-f(\mc{W})=-\mrm{sgn}(p)F^{p}(\mc{W})=\~F^{-p}(\~{\mc{W}})}
satisfies the assumptions of \cref{Sphere}, which in particular implies
\eq{\del_t\br{\Phi t^{\fr{-p}{-p+1}}}\geq 0.}
}
\subsection*{De Sitter space}
For flows of spacelike hypersurfaces in Lorentzian manifolds of nonvanishing curvature the second line of \eqref{Ev-q} can behave rather differently, since $\nu$ is timelike. Hence, in the De Sitter space of constant sectional curvature $K_{N}=1$ we obtain a similar result as in the sphere, but only for flow with principal curvatures bounded be $1$. This is equivalent to convexity by horospheres for the dual hypersurfaces in the hyperbolic space and hence seems to be a natural assumption for flow in the De Sitter space.
\Theo{thm}{DeSitter}{
Let $N=\mathbb{S}^{n,1}$ and $F$ be a monotone, convex and $1$-homogeneous curvature function. Let $0<p\leq 1$ and set $f=F^p$. Then for any solution $x$ of \eqref{Flow} that the condition $0<\kappa_i\leq 1$ is always satisfied on $[0,T^{\ast})$ there holds
\eq{tu+\fr{p}{p+1}\geq 0.}
}
\pf{
Let us put $X=\dot{x}+\sigma f\nu.$ We start with the following observations:
\begin{align*}
\bar{R}(\dot{x},\nu,\dot{x},\nu)&=-\langle \dot{x},\dot{x}\rangle-\langle \dot{x},\nu\rangle^2\\
&=f^2-\langle X,X\rangle-f^2\leq 0,
\end{align*}
\begin{align*}
\bar{R}(\dot{x},x_{;i},\dot{x},x_{;i})&=\langle \dot{x},\dot{x}\rangle g_{ii}-\langle \dot{x},x_{;i}\rangle^2\\
&=-f^2g_{ii}+\langle X,X\rangle g_{ii}-
\langle X,x_{;i}\rangle^2\geq -f^2g_{ii}.
\end{align*}
\eqref{Ev-q} with $\mc{B}=0$, these last two inequalities and
$\Lambda_{ij}=fg_{ij}$
imply that $q$ satisfies
\eq{\label{xx}\mathcal{L}q\geq&\br{\fr{p+1}{p}u-\fr{4}{p}f^{ij}g_{ij}}q+2t\left(\fr{1}{p}\br{f^{ij}g_{ij}}^2-pf^2\right)+\fr{4}{p+1}f^{ij}g_{ij}.}
On the other hand, since $0<\kappa_i\leq 1$, we have
\[pf=\sum \frac{\partial f}{\partial \kappa_i}\kappa_i\leq f^{ij}g_{ij}\Rightarrow \fr{1}{p}\br{f^{ij}g_{ij}}^2-pf^2\geq 0.\]
The result now follows from the maximum principle.
}
\Theo{rem}{}{
We cannot expect to obtain a Harnack estimate with a bonus term  for mean curvature flow in the De Sitter space as in the spherical case. To see that, we will look at  ancient solutions with $0<\kappa_i\leq 1$ to the mean curvature flow.

The evolution equation of $H$ is given by
\[\partial_tH=\Delta H+T\ast\nabla H-|A|^2H+nH.\]
If there was a Harnack inequality for mean curvature flow of the following form
\[\partial_tH-nH+\frac{H}{2t}\geq 0,\]
then for an ancient solution we would have
$\partial_tH-nH\geq 0.$
So evolution equation of $H$ would yield
$\Delta H+T\ast\nabla H-|A|^2H\geq 0;$ therefore, $H(\cdot,t)=0.$
}
%\Theo{thm}{MCF with bonus}{Let $N=\mathbb{S}^{n,1}$ and $f=H$. Then under \eqref{Flow} we have
%\[\partial_tH-nH+\frac{H}{2t}\geq 0.\]}
%\pf{Define $q:=w-nt$. Using (\ref{xx}) and ignoring its last two terms (since their sum is non-negative), we deduce that for $f=H:$
%\[\dot{q}\geq \Delta {q}+T\ast\n q+2\left(u-n\right)q.\]
%Hence by the maximum principle, $q$ remains positive. This implies that
%\[\partial_t H-nH+\frac{H}{2t}\geq 0.\]
%}

With precisely the same proof as for \cref{Sphere}, we obtain, using \eqref{DualHyperbolic} and \cref{DeSitter}, the following pseudo-Harnack inequality for expanding flows of the hyperbolic space, which is to our knowledge the first such inequality for hypersurface flows in the hyperbolic space:
\Theo{thm}{Hyperbolic}{Let $N=\H^{n+1}$ and $F$ be a monotone, 1-homogeneous and inverse convex curvature function. If $-1\leq p<0$, then any horoconvex solution to \eqref{NormalFlow} with speed $f=-F^{p}$
satisfies
\[\del_t\br{ft^{\fr{p}{p-1}}}\leq 0.\]
}
\pf{The speed of the dual flow \eqref{DualHyperbolic} is
\[\Phi(\~{\mc{W}})=-f(\mc{W})=-\mrm{sgn}(p)F^p(\mc{W})=-\mrm{sgn}(p)F^p(\~{\mc{W}}^{-1})=-\mrm{sgn}(p)\fr{1}{\~F^p(\~{\mc{W}})}\]
and if $-1\leq p<0$, then
\[\Phi(\~{\mc{W}})=\mrm{sgn}(-p)\~F^{-p}(\~{\mc{W}}).\]
Thus for $-1\leq p<0,$ the assumptions of \cref{DeSitter} are satisfied with $0<-p\leq 1$; therefore,
\[\del_t\br{\Phi t^{\fr{-p}{-p+1}}}\geq 0\Rightarrow \del_t\br{f t^{\fr{p}{p-1}}}\leq 0.\]
}
\section{Preserving convexity/horoconvexity}
In our main theorems we always assume that we have a strictly convex flow, in fact, otherwise we can not even write down the reparametrization \eqref{Flow}. In this section we justify this assumption by showing that in many cases the strict convexity of the flow hypersurfaces is preserved under \eqref{Flow}. We start with the De Sitter space.
\Theo{prop}{PresConvDeSitter}{
Let $N=\S^{n,1}$ and $F$ be a monotone, convex and $1$-homogeneous. Let $0<p\leq 1$ and set $f=F^p$. Let $M_0\sub N$ be a strictly convex, closed and spacelike hypersurface, such that the Beltrami point of the dual hypersurface $\~M_0$ lies in the interior of the convex body enclosed by $\~M_0$. Then the strict convexity is preserved under the flow
\eq{\label{PresConvDeSitter-1}\dot{x}=f\nu.}
}
\pf{
We use the dual inverse curvature flow in the hyperbolic space $\H^{n+1}$. Using \Cref{gauss_duality} we obtain that up to reparametrization the dual flow, denoted by $\~x$ evolves according to
\eq{\label{PresConvDeSitter-2}\dot{\~x}=\fr{1}{\~F^p}\hat{\n}u,\quad 0<p\leq 1,}
where $\hat{\n}u$ is the outward unit normal to the flow hypersurfaces in $\H^{n+1}$. Now $\~M_0$ is starshaped with respect to the origin in $\H^{n+1}$ (the Beltrami point) and $\~F$ is the inverse of $F$. Hence $\~F$ is concave, compare \Cref{subsec:bg_speed}. Precisely under these assumptions on $\~F$ and $p$ we have shown in \cite[Thm.~1.2]{Scheuer:05/2015}, that the flow \eqref{PresConvDeSitter-2} with initial hypersurface $\~M_0$ exists for all times (regardless of preserved convexity!).
Hence, at a first time $t_0<\8$ where \eqref{PresConvDeSitter-1} loses convexity, the principal curvatures of $\~M_{t_0}$ are still bounded, which is a contradiction,\footnote{$M_0$ is strictly convex $\Rightarrow\~M_0$ is strictly convex $\Rightarrow\~M_0$ is starshaped (see, e.g., \cite[Section 3.2]{Gerhardt:/2006} for the last step)  $\Rightarrow\~\kappa_i\leq c_{\tilde{M}_0},~\forall t>0$  $\Rightarrow\kappa_i$ cannot go to zero as $t$ approaches $t_0$} so such a time $t_0$ does not exist.
}

In order to show that the condition $\k_i\leq 1$ is preserved we need an additional assumption on the speed, namely that $f=F^p$ is also concave.
Then we get:

\Theo{thm}{PresCurv}{
Let $N=\S^{n,1}$ and $F$ be a monotone, convex and $1$-homogeneous. Let $0<p\leq 1$, set $f=F^p$ and suppose that $f$ is concave. Let $M_0\sub N$ be a strictly convex, closed and spacelike hypersurface, such that $\k_i\leq 1$ for $1\leq i\leq n$ and the Beltrami point of the dual hypersurface $\~M_0$ lies in the interior of the convex body enclosed by $\~M_0$. Then the condition $\k_i\leq 1$ is preserved under the flow
\eq{\label{PresCurv-1}\dot{x}=f\nu.}
}
\pf{
Due to \cite[Lemma~2.3.1, Lemma~2.4.3]{Gerhardt:/2006}, along \eqref{PresCurv-1} the tensor
\eq{S_{ij}=h_{ij}-g_{ij}}
satisfies the evolution equation
\eq{\dot{S}_{ij}-f^{kl}S_{ij;kl}=&-f^{kl}h_{kr}h^r_lh_{ij}+(1+p)f(h^k_ih_{kj}+g_{ij})-f^{kl}g_{kl}h_{ij}-2fh_{ij}\\
	&+f^{kl,rs}h_{kl;i}h_{rs;j}\\
                        \equiv& N_{ij}.
}
We use Hamilton's tensor maximum principle, \cite[Thm.~9.1]{Hamilton:/1982}. At a unit null eigenvector $v$ of $S_{ij}$ we obtain, also using the concavity of $f$,
\eq{N_{ij}v^iv^j&\leq -f^{kl}h_{kr}h^r_l+2f+2pf-f^{kl}g_{kl}-2f\\
				&=-f^{kl}(h_{kr}h^r_l-2h_{kl}+g_{kl})\\
                &\leq 0.}
Hence \cite[Thm.~9.1]{Hamilton:/1982} is applicable and we conclude that $S_{ij}$ remains non-positive.
}
\section{Cross curvature flow}\label{cross}
Let $(M^3,g)$ be a Riemannian 3-manifold with negative sectional curvature. The cross curvature tensor is defined by
\[c_{ij}:=(E^{-1})_{ij}\det E=\frac{1}{2}\mu^{ipq}\mu^{jrs}E_{pr}E_{qs}=\frac{1}{8}\mu^{pqk}\mu^{rsl}R_{ilpq}R_{kjrs},\]
where $E_{ij}:=R_{ij}-\frac{1}{2}Rg_{ij}$ is the Einstein tensor, $R_{ijkl}$ is the Riemann curvature tensor, $R_{ij}$ is the Ricci curvature tensor and $R$ is the scalar curvature, $\det E:=\det E_{ij}/\det g_{ij}$ and $\mu^{ijk}$ are the component of the volume form.

A one parameter family of 3-manifolds $(M,g(t))$ with negative sectional curvature is a solution of the XCF if
\[\partial_tg_{ij}=2c_{ij}.\]
Now suppose metrics are locally isometrically embeddable in Minkowski space $\mathbb{R}^{3,1}$. The following observation is due to Andrews, which recently appeared in \cite{AndrewsChenFangMcCoy:/2015}.

Recall that the Gauss equation in $\mathbb{R}^{3,1}$ reads
\[
R_{ijkl} = -(h_{ik}h_{jl} - h_{il}h_{jk}).
\]
Tracing with respect to $g^{ik}$ gives
\[
R_{jl} = -(Hh_{jl} - h^k_lh_{jk}), \quad R = -(H^2 - |A|^2),
\]
where $|A|^2=g^{ik}g^{jl}h_{ij}h_{kl}.$
Thus we have
\[
E_{ij} = H\left(\frac{H}{2}g_{ij} - h_{ij}\right) + \left(h^k_ih_{kj} - \frac{1}{2}|A|^2g_{ij}\right).
\]
In an orthonormal frame which diagonalizes the second fundamental form, we get for $i=1$:
\[
\begin{split}
E_{11} &= \frac{1}{2}\left(H^2 - |A|^2\right) + h^1_1 h_{11} - Hh_{11} \\
&= \frac{1}{2}\left(2(h_{11}h_{22} + h_{11}h_{33} + h_{22}h_{33})\right) + h_{11}^2 - \left(h_{11} + h_{22} + h_{33}\right) h_{11} \\
&= h_{22}h_{33},
\end{split}
\]
and similarly for $i=2,3$. That is,
\[
E = \begin{pmatrix}
\kappa_2 \kappa_3 & 0 & 0 \\
0 & \kappa_1\kappa_3 & 0 \\
0 & 0 & \kappa_1\kappa_2
\end{pmatrix}
\]
where $\kappa_i$ denote the principal curvatures. In particular, $\det E = K^2,$ where $K$ is the Gauss curvature. If $M$ is strictly convex, then $E$ is positive definite, hence invertible. In this case, the cross curvature tensor is
\[
c_{ij} = ({\det} E) (E^{-1})_{ij} = \begin{pmatrix}
\kappa_1^2 \kappa_2 \kappa_3 & 0 & 0 \\
0 & \kappa_1\kappa_2^2 \kappa_3 & 0 \\
0 & 0 & \kappa_1\kappa_2\kappa_3^2
\end{pmatrix} = Kh_{ij}.
\]
Now the uniqueness result of Buckland \cite{Buckland:/2006} shows that $(M,g(t))$ is a solution of (\ref{FlowStandard}) with $N=\mathbb{R}^{3,1}$, $f=K$.

The Harnack inequality for the cross curvature flow for metrics that are locally isometrically embeddable in Minkowski space $\mathbb{R}^{3,1}$ now follows from \Cref{Euclidean}:
\begin{align}\label{cross harnack}
\partial_t \sqrt{{\det} E} - \frac{1}{\sqrt{{\det}E}}E^{ij} \nabla_i \left(\sqrt{{\det}E}\right) \nabla_j \left(\sqrt{{\det}E}\right) + \frac{3}{4 t} \sqrt{{\det}E} \geq 0.
\end{align}


\bibliographystyle{amsplain}
\bibliography{Bibliography}


\end{document}

